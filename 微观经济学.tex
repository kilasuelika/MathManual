\chapter{微观经济学}
本章节主要以田国强的《高级微观经济学》为基础编写。
\section{微观经济学的数学基础}
\subsection{序关系与偏好理论}

\subsection{微观经济学中的函数}
\subsubsection{效用函数与生产函数}
效用函数与生产函数基本上是一回事,“效用”就相当于物品“生产”出的效用。只是在一般性质上有差异。生产函数一般总是能作为效用函数的。

\paragraph*{生产函数的性质}连续生产函数一般满足以下性质:
\begin{enumerate}
\item $f>0$:产量为正。
\item $f'>0$:产量是严格增函数。
\item $f''<0$:边际报酬递减。
\end{enumerate}

以下如果没有说明,则默认都可以用来作为效用或者生产函数。

\paragraph*{单变量效用函数}比如消费、某种商品的效用函数:
\begin{empheq}{align}
u(C)&=C-a\frac{C^2}{2}\mtag{二次型}\\
u(C)&=\ln(C)\mtag{对数型}\\
u(C)&=C^{\alpha},\quad 0<\alpha\leq 1\mtag{指数型}
\end{empheq}


\paragraph*{单变量效用函数的叠加}对单变量效用函数进行叠加可以得到多变量的效用函数:
\begin{empheq}{align}
u(\bm{C})&=\sum w_iU_i(C_i),\quad w_i>0,\sum w_i=1\mtag{加权和}\\
u(\bm{C})&=\prod u_i(C_i)^{\alpha_i},\quad \alpha_i>0,\sum \alpha_i=1\mtag{乘积}
\end{empheq}
以下说明一些常用的例子:
\begin{description}
\item[柯布——道格拉斯函数] 指数型的乘积:$u(C_1,C_2)=C_1^{\alpha}C_2^{1-\alpha}$。$u(C_1,C_2,C_3)=C_1^{\alpha}(C_2C_3)^{1-\alpha}$,这里是说$C_1,C_2$需要联合起来发挥作用。
\end{description}


\subsection{微观经济学中的导出量}
\subsubsection{弹性}
给定一个函数$y=f(x)$,定义
\begin{empheq}{equation}
E_{y,x}=\odv{y}{x}\frac{x}{y}=f'(x)\frac{x}{y}
\end{empheq}
为$y$对$x$的弹性。

如果$x$增加$a$比例,则当$x$较小时,比如1\%,$y$会变动$aE$。当$x$较大时,$y$的近似变动量为:
$$(1+a)^E-1$$
这只是一个近似,是通过固定$E$,解ODE得到的。

\section{个体决策}

\section{博弈论}
\subsection{完全信息静态博弈}

\subsection{完全信息动态博弈}

\subsection{不完全信息静态博弈}

\subsection{不完全信息动态博弈}

\subsection{重复博弈}

\subsection{合作博弈}

\section{市场理论}

\section{市场失灵}
\subsection{外部性}

\subsection{公共产品}

\subsection{交易成本}


\section{拍卖}

\section{福利与一般均衡理论}
\subsection{竞争均衡}

\subsection{福利经济学基本定理}


\subsection{不确定信息下的一般均衡理论}

\section{机制设计}
\subsection{委托-代理理论}
\subsection{完全信息下的一般机制设计}

\subsection{不完全信息下的一般机制设计}

\section{匹配}
