\chapter{流体力学}
\section{物理量}
\subsection{压强}
压强有很多种:
\begin{description}
\item[静态压强Static Pressure] 忽略速度效应。
\item[Gauage Pressure] 以某个参考点为零点的压强,通常是标准大气压。
\item[动态压强Dynamic Pressure] 
\item[总压强Total Pressure] 
\item[绝对压强Absolute Pressure] 以0Pa为零点的压强。仿真时推荐用这个。
\end{description}
\section{原理}

\section{流体力学基本方程}
\subsection{Navier-Stokes方程}
\subsubsection{方程形式}
NS方程的一般形式
\begin{empheq}{align}
\pdv{\rho}{t}+\vdiv (\rho \bm{U})&=0 \mtag{质量守恒、连续性方程}\label{ns-1}\\
\pdv{\rho \bm{U}}{t}+\vdiv (\rho \bm{U}\bm{U})&=-\nabla p+\vdiv\bm{\tau}+\bm{F} \mtag{动量方程} \label{ns-2}
\end{empheq}
其中$\rho$为标量密度场,$\bm{U}$为向量速度场,$p$为标量压力场,$\bm{\tau}$为矩阵剪切应力场,$\bm{F}$为向量外力场。因此第一个方程为标量方程,而第二个方程为向量方程,对应$n$个方程($n$为维度)。

式\eqref{ns-2}的左边在有的地方也写成:
\begin{empheq}{equation}
\rho\left(\pdv{\bm{U}}{t}+\bm{U}\cdot \nabla \bm{U}\right)
\end{empheq}

这是因为依据\ref{sec-vector-analysis},式\eqref{ns-2}的左边
\begin{empheq}{align}
\pdv{\rho \bm{U}}{t}+\vdiv (\rho \bm{U}\bm{U})&=\bm{U}\pdv{\rho}{t}+\rho\pdv{\bm{U}}{\bm{U}}+\bU\nabla\cdot(\rho\bU)+\rho\bU\cdot\nabla \bU\\
&=\bU\underbrace{\left(\pdv{\rho}{t}+\vdiv (\rho \bm{U})\right)}_{\text{连续性方程}}+\rho\left(\pdv{\bm{U}}{t}+\bm{U}\cdot \nabla \bm{U}\right)\\
&=\rho\left(\pdv{\bm{U}}{t}+\bm{U}\cdot \nabla \bm{U}\right)
\end{empheq}

\subsubsection{剪切应力场}

\paragraph*{牛顿流体}如果流动过程中流体层间所产生的剪应力与法向速度梯度成正比,就称为牛顿流体,有
$$\bm{\tau}=2\mu\bm{S}=\mu(\nabla \bU+\nabla \bU^T)$$
$\mu$为粘度。代入式\eqref{ns-2}有
\begin{empheq}{equation}
\pdv{\rho \bm{U}}{t}+\vdiv (\rho \bm{U}\bm{U})=-\nabla p+\vdiv(\mu (\nabla \bU+\nabla \bU^T))+\bm{F}
\end{empheq}

如果$\bm{\tau}$是常量,则
$$\nabla \cdot \bm{\tau}=\mu(\Delta \bm{U}+\nabla\cdot(\nabla \bm{U}))$$
\paragraph*{不可压缩流}对于不可压缩流,不可压缩时密度为常量$\rho_0$,则
$$\vdiv \bU=0$$
于是牛顿流体中:
$$\nabla \cdot \bm{\tau}=\mu\Delta \bU$$
\paragraph*{可压缩流体}剪切应力场在牛顿流体上还要加上一个量:
$$\bm{\tau}=\mu (\nabla \bU+\nabla \bU^T)-\frac{2}{3}\mu(\vdiv \bU)I$$

\paragraph*{理想流体}粘度的引入会使得求解变得困难,有些问题中粘度显示不出来,比如均匀流动的情形,此时假定粘度为0,称为理想流体。此时$\bm{\tau}=\bm{0}$。
\subsubsection{特殊情形}
不考虑外力。
\paragraph*{稳态}密度不随时间变化,消去时间项:
\begin{empheq}{align}
\vdiv (\rho \bm{U})&=0 \label{stable-ns-1}\\
\vdiv (\rho \bm{U}\bm{U})&=-\nabla p+\vdiv\bm{\tau} \label{stable-ns-2}
\end{empheq}
第一个方程表示$\rho\bU$为无源场。

\paragraph*{不可压缩流体、稳态}
\begin{empheq}{align}
\vdiv \bm{U}&=0 \\
-\mu\Delta \bU+\nabla p&=-\rho_0 \bU\cdot\nabla\bU
\end{empheq}
第二个方程的右边为惯性力,在低Re数的情况下,右边的项较小,因此还可以忽略右边,此时有
\begin{empheq}{align}
\vdiv \bm{U}&=0 \\
-\mu\Delta \bU+\nabla p&=0
\end{empheq}
\subsubsection{边界条件}
