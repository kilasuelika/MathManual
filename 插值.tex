\chapter{插值}
从广泛的意义上说,回归也算是插值,但一般来说,插值特指的是在样本点上误差为0,同时要尽可能简单。

\section{一维插值}
\subsection{分段线性插值}
就是最简单的插值,给定$n+1$个样本点:$(x_i,y_i)$,要求$\forall i<j:x_i<x_j$,即点有序。分段线性插值就是用$n$条线段依次连接每一点。假设目标点就$(x,y)$,$x$已知,且$x_k\leq x\leq x_{k+1}$。则有
\begin{empheq}{align*}
\lambda x_k+(1-\lambda) x_{k+1}&=x\\
\lambda y_k+(1-\lambda) y_{k+1}&=y
\end{empheq}
由第1式可得:
\begin{empheq}[box=\myalgo]{equation}
\lambda=\frac{x_{k+1}-x}{x_{k+1}-x_k}
\end{empheq}
代入第2式有
\begin{empheq}[box=\myalgo]{equation}
y=y_{k+1}-\lambda (y_{k+1}-y_k)
\end{empheq}

那么怎么来得到$k$,可以用循环判断,也可以用二分搜索,还可以用向量化编程:

\texttt{(x>=vx) \&\& (x<xv)}

这样可以得到boolean下标,表示$x$是否属于对应区间,找到第一个为\texttt{True}的下标就可以了。

\subsection{拉格朗日多项式}
\subsubsection{方法}
\paragraph*{基本算法}
给定$n+1$个样本点,不要求有序,拉格朗日多项式法形如
\begin{align}
L_k(x)&\coloneqq \prod_{j\neq k}^{n}\frac{x-x_j}{x_k-x_j}\\
&=\frac{(x-x_0)\cdots(x-x_{n})}{(x_k-x_0)\cdots\widehat{(x_k-x_k)}\cdots(x_k-x_n)}\\
I_n(x)&=\sum_{k=0}^{n}y_kL_k(x)
\end{align}
$I_n(x)$就是目标的插值。$L_k$就是$n$次拉格朗日多项式,它满足:$L_k(x_k)=1,\forall j\neq k,L_k(x_j)=0$。还有一个重要的性质是:
\begin{empheq}[box=\mymath]{equation}\label{interpolation-lagrange-all-1}
\sum_{k=0}^{n} L_k(x)=\tilde{I}_n(x)=\sum_{k=0}^{n}\prod_{j\neq k} \frac{x-x_j}{x_k-x_j}=1
\end{empheq}
从插值的意义上说,这是显然的,相当于$\forall i, y_i=1$时的插值多项式$\tilde{I}_n(x)$,则必然所有插值均为1。实际上根据代数学基本定理\ref{basic-theorem-in-algebra},给定零点,可以唯一确定一个最高次系数为1的多项式。此处$\tilde{I}_n(x)$恰为一个$n$次多项式,$\tilde{I}_n(x)-1$的零点有$n+1$个,为$x_k$。再取常多项式$p(x)=0$,$x_k$也是它的零点,于是根据唯一性有$\tilde{I}_n(x)-1=p(x)=0$。


$I_n(x)$还可以表示为:
\begin{align}
I_n(x)=\left\{\sum_{k=0}^{n}y_k\inv{x-x_k}\underbrace{\prod_{j\neq k}^{n}\inv{x_k-x_j}}_{\lambda_k}\right\}\prod_{j=0}^{n}(x-x_j)
\end{align}
如果记$\mu_k\coloneqq \frac{\lambda_k}{x-x_k}$为插值权重,则有
\begin{equation*}
I_n(x)=\left\{\sum_{k=0}^{n}\mu_iy_i\right\}\prod_{j=0}^{n}(x-x_j)
\end{equation*}
根据\cref{interpolation-lagrange-all-1},有
$$\prod_{j=0}^{n}(x-x_j)=\inv{\sum_{k=0}^n\mu_k}$$
于是有
$$I_n(x)=\frac{\sum_{k=0}^{n}\mu_ky_k}{\sum_{k=0}^n\mu_k}$$
这也叫重心公式。

最终得到向量化的插值算法:
\begin{empheq}[box=\myalgo]{align}
\lambda_k&=\prod_{j\neq k}^{n}\inv{x_k-x_j}\\
\bmu&=\frac{\blambda}{x-\bx}\\
\alpha&=\sum_{k=0}^{n}\mu_k\\
I_n(x)&=\frac{\bmu\cdot\by}{\alpha}
\end{empheq}
这个算法的复杂度为$O(n^2)$。

假设增加一个样本点$(x_{n+1},y_{n+1})$,则首先更新$\lambda$:
\begin{empheq}[box=\myalgo]{align}
\lambda_{k\leq n}&=\frac{\lambda_{k}}{x_k-x_{n+1}}\\
\lambda_{n+1}&=\prod_{j=0}^{n}\inv{x_{n+1}-x_j}
\end{empheq}
再按之前的式子更新$\bmu,\alpha,I_{n+1}(x)$,复杂度为$O(n)$。

\paragraph*{直观的构造}我们希望寻找一组多项式构成的基$f_k(x)$,以满足:
$$I_n(x)=\sum_{k=0}^{n}f_k(x)y_k$$
希望$f_k(x)$满足$f_k(x_k)=1,\forall k\neq j:f_k(x_j)=0$即在其它点处为0,在本点处为1。在其它点处为0,很容易构造,那就是
$$g_k(x)=(x-x_0)\cdots \widehat{x-x_k}\cdots(x-x_n)$$
要在本点处为1,就要标准化一下:
$$f_k(x)=\frac{g_k(x)}{g_k(x_k)}$$
同样可以得到之前的结果。

但仅凭这样并不能得到\cref{interpolation-lagrange-all-1},因为这里只考虑了节点上成立,\cref{interpolation-lagrange-all-1}是对任意$x$均成立的。

\paragraph*{线性代数的构造}我们希望求一个$n$次多项式(有$n+1$个未决系数)来拟合$n+1$个点,等价于解方程组:
\begin{empheq}{equation}
\by=A\bm{a}\implies \bm{a}=A^{-1}\by
\end{empheq}
式中$A$为范德蒙德矩阵:
\begin{empheq}{equation}
A=\begin{bmatrix}
1 & x_0 & x_0^2 &\cdots & x_0^n\\
1 & x_0 & x_1^2 &\cdots & x_1^n\\
1 & \cdots & & & \\
1 & x_n & x_n^2 &\cdots & x_n^n
\end{bmatrix}
\end{empheq}
则目标插值为
\begin{empheq}[box=\myalgo]{equation}
I_n(x)=\begin{bmatrix}
1 & x & x^2 &\cdots & x^n
\end{bmatrix}A^{-1}\by
\end{empheq}
可以证明这种方法与之前的拉格朗日插值多项式给出的结果相同,不过之前的算法要略微快一些。


\paragraph*{实例}


\subsubsection{理解}

\subsection{样条插值}
\subsubsection{一般理论}

