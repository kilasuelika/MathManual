\chapter{形式化系统}

\section{EBNF}
EBNF(Extended Backus–Naur form)可以用来描述上下文无关语法。
\subsection{语法元素}
一个语法需要两类元素,第一是terminal,即最小的元素,第二是规则,描述了元素间的转换关系。定义terminal与规则可以使用以下一些基本操作:
\begin{longtable}{cc}
	\toprule
	含义&	符号\\
	\midrule
	definition	&$=$\\
	concatenation&	$,$\\
	termination	&$;$\\
	alternation	&$|$\\
	optional	&$[\ \cdots\ ]$\\
	repetition	&$\{\ \cdots\ \}$\\
	grouping &	$(\ \cdots\ )$\\
	terminal string	&$"\ \cdots\ "$\\
	terminal string	&$'\ \cdots\ '$\\
	comment	&$(*\ \cdots\ *)$\\
	special sequence	&$?\ \cdots\ ?$\\
	exception&	$-$\\
	\bottomrule
\end{longtable}

\subsection{案例}

\section{CCS}

\subsection{基本技巧}
\begin{enumerate}
  \item 描述一个单元需要多种输入才能进入下一个状态.由于原版的CCS的交互只涉及2个个体,没有3个个体同时交互,那么就需要引入多个状态.比如需要2种资源,就用4个状态:空、等待资源1、等待资源2、准备完成.空状态得到资源1就进入等待资源2,再得到资源2则准备完成.

      但有时如果很明确就是三个个体交互,那么应该也可以让一个个体接受多个输入,其它个体输出,一次完成多个个体交互.
  \item
\end{enumerate} 

