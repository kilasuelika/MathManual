\chapter{矩阵代数}

\section{基本矩阵运算}
%\subsection{公式}
\subsection{矩阵乘法与基本算子}
\paragraph*{核心原理}
\begin{empheq}{align*}
	\bm{x}^T\bm{y}&=\bm{y}^T\bm{x}=\text{trace}(\bm{x}\bm{y}^T)\\
	\by^T\bx\bx^T\by&=(\bx^T\by)^2=(\by^T\bx)^2\\
    \bx^TA\by&=\trace(\by\bx^TA)\\
	\text{trace}(AB)&=\text{trace}(BA)\\
	\trace(ABC)&=\trace(BCA)=\trace(CBA)\\
	\det(A^{-1})&=\det(A)^{-1}\\
	\det(e^A)&=e^{\trace(A)}\\
	\det (I_m+AB^T)&=\det (I_n+A^TB)\\
	\det(I+\bm{a}\bm{b}^T)&=1+\bm{a}^T\bm{b}\\
	\det (I+A)&=1+\det A+\text{trace}(A)\\
	\det(I+\varepsilon A)&\approx 1+\det(A)+\varepsilon\trace(A)+\inv{2}\varepsilon^2\trace(A)^2-\inv{2}\varepsilon^2\trace(A^2)\\	\det\left(A^{-1}+A^{-1}BA\right)&=\det\left(A^{-1}A^{-1}A++A^{-1}BA\right)=\det\left(A^{-1}+B\right)\\
\det \begin{bmatrix}
A&B\\C&D\\
\end{bmatrix}&=\det (A)\det(D-CA^{-1}B)\\
 &=\det (D)\det(A-BD^{-1}C)
\end{empheq}
以上假定$\diag$是列向量。

矩阵运算如果在运算过程中产生了标量,那么不一定满足结合律,比如:
$$\bx^T\bx\bx\neq \bx^T(\bx\bx)$$

右边是没有定义的.但是假如没有生成标量,那么就是结合的.这是因为标量乘以矩阵,实际上是两个元素,无法进一步化简了;而多个矩阵乘起来,最终得到一个矩阵,只有一个元素.

此外还要强调:
\begin{empheq}{align*}
ABA+ACA&=A(B+C)A\\
ACB+DCE&\neq (A+D)C(B+E)
\end{empheq}

后面这个地方比较容易出错。

矩阵运算中,要注意几点:

\begin{enumerate}
	\item 假如包含标量,可以放到前面或者后面、可以转置.标量非常重要的,要注意区分.
	\item 注意一些基本的量,可能隐含一些潜在的关系,比如求值使两个向量相减为0,则向量是平行的,正交回归是一例.
	\item 在式子中插入一个单位项$I=AA^{-1}=A^{-1}A$用于构造二次型,在多元高斯分布的计算中比较常用。
	\item 掌握一些改写方程的技巧:
	
		\begin{empheq}{align*}
			1+\bm{\theta}^T\bm{\theta}&=\begin{bmatrix}
				\bm{\theta}^T&-1
			\end{bmatrix} \begin{bmatrix}
				\bm{\theta}\\-1
			\end{bmatrix}\\
			\by-X\bm{\theta}&=\begin{bmatrix}
				X&y
			\end{bmatrix} \begin{bmatrix}
				\bm{\theta}\\-1
			\end{bmatrix}
		\end{empheq}
\end{enumerate}

\begin{note}
\begin{enumerate}
\item 证明矩阵恒等式$\det(e^A)=e^{\trace A}$。考虑函数$f(t)=\det(e^{tA})$,求导有$f'(t)=f(t)\trace A$,这里的求导参考矩阵微分\eqref{derivative-det},解方程有$f(t)=Ce^{t\trace A}$,配合初始条件$f(0)=1$,就得到原式。
\end{enumerate}

\end{note}

\subsection{矩阵逆}
\paragraph*{原理}
\begin{empheq}{align}
(AB)^{-1}&=B^{-1}A^{-1}\\
(A+BD^{-1}C)^{-1}&=A^{-1}-A^{-1}B(D+CA^{-1}B)^{-1}CA^{-1}\mtag{Woodbury}\label{Woodbury}\\
\left(A+uv^T\right)^{-1}&=A^{-1}-\frac{A^{-1}uv^TA^{-1}}{1+v^TA^{-1}u}\mtag{Sherman-Morrison}\\
(A+UV^T)^{-1}&=A^{-1}-A^{-1}U(I+V^TA^{-1}U)^{-1}V^TA^{-1}\mtag{Shermon-Morrison-Woodbury}\\
(I+B)^{-1}&=(I+IBI)^{-1}\\
	&=I-(B^{-1}+I)^{-1}\\
	&=I-B(I+B)^{-1}\\
(I+AB)^{-1}A&=A(I+BA)^{-1}\\
(A^{-1}+B^TC^{-1}B)^{-1}B^TC^{-1}&=AB^T(BAB^T+C)^{-1}
\end{empheq}

\begin{empheq}[left=\empheqlbrace]{align*}
\begin{bmatrix}
	A&D\\
	C&B
\end{bmatrix}^{-1}&=
\begin{bmatrix}
	A^{-1}+E\Delta^{-1}F & -E\Delta^{-1}\\
	-\Delta^{-1}F & \Delta^{-1}
\end{bmatrix}\mtag{分块矩阵逆I}\\
\Delta&\coloneqq B-CA^{-1}D\\
E&\coloneqq A^{-1}D\\
F&\coloneqq CA^{-1}
\end{empheq}

\begin{empheq}{align}
\begin{bmatrix}
	A&\bm{0}\\
	C&B
\end{bmatrix}^{-1}&=
\begin{bmatrix}
	A^{-1}&\bm{0}\\
	-B^{-1}CA^{-1}&B^{-1}
\end{bmatrix}\\
\begin{bmatrix}
	A&C\\
	\bm{0}&B
\end{bmatrix}^{-1}&=
\begin{bmatrix}
	A^{-1}&-A^{-1}CB^{-1}\\
	\bm{0}&B^{-1}
\end{bmatrix}
\end{empheq}

\begin{empheq}[left=\empheqlbrace]{align*}
	\begin{bmatrix}
		A&D\\
		C&B
	\end{bmatrix}^{-1}&=
	\begin{bmatrix}
		\Delta^{-1} & -\Delta^{-1}E\\
		-F\Delta^{-1} & B^{-1}+F\Delta^{-1}E
	\end{bmatrix}\mtag{分块矩阵逆II}\\
	\Delta&\coloneqq A-DB^{-1}C\\
	E&\coloneqq DB^{-1}\\
	F&\coloneqq B^{-1}C
\end{empheq}

\paragraph*{Shermon-Morrison-Woodbury定理的证明}基本思路是构造矩阵方程组,再用消元法。

取辅助矩阵
$$H=\begin{bmatrix}
A & U\\
-V^T&I
\end{bmatrix}$$
显然有
\begin{empheq}{align*}
\begin{bmatrix}
I&0\\
V^TA^{-1}&I
\end{bmatrix}H&=\begin{bmatrix}
A & U\\
0&I+V^TA^{-1}U
\end{bmatrix}\\
\begin{bmatrix}
I&-U\\
0&I
\end{bmatrix}H&=\begin{bmatrix}
A+UV^T & 0\\
-V^T&I
\end{bmatrix}
\end{empheq}
两式结合有:
\begin{equation}\label{SMW-proof-core-eq}
\begin{bmatrix}
A+UV^T & 0\\
-V^T&I
\end{bmatrix}=\begin{bmatrix}
I&-U\\
0&I
\end{bmatrix}\begin{bmatrix}
I&0\\
V^TA^{-1}&I
\end{bmatrix}^{-1}\begin{bmatrix}
A & U\\
0&I+V^TA^{-1}U
\end{bmatrix}
\end{equation}

然后根据矩阵逆有
\begin{empheq}{align*}
\begin{bmatrix}
A+UV^T & 0\\
-V^T&I
\end{bmatrix}^{-1}=\begin{bmatrix}
(A+UV^T)^{-1} & 0\\
\star&I
\end{bmatrix}&
\end{empheq}
左边再按\eqref{SMW-proof-core-eq}展开再比较系数即可,注意上式中的$\star$不必计算。

\paragraph*{Moore-Penrose伪逆}用来给非方阵求逆。假如$A\in\mathbb{R}^{n\times d}$,则$A^+\in\mathbb{d\times n}$,即形状恰好互为转置。

\begin{definition}[Moore-Penrose伪逆]\label{moore-penrose-pseudo-inv}
对于任意矩阵$A$,定义Moore-Penrose伪逆为满足以下条件的唯一矩阵:
\begin{enumerate}
\item $AA^+A=A$
\item $A^+AA^+=A$
\item $(AA^+)^H=AA^+$,即$AA^+$为Hermitian。
\item  $(A^+A)^H=A^+A$
\end{enumerate}

它满足的性质有:
\begin{enumerate}
\item 如果$A$为实矩阵,则$A^+$也是。
\item 如果$A$可逆,则$A^+=A^{-1}$。
\end{enumerate}
\end{definition}
\subsection{多项式}
\subsubsection{几何级数}
\begin{theorem}[矩阵几何级数的收敛]\label{matrix-geometric-series}
$A$为方阵,则:
\begin{enumerate}
\item $\sum_{k=0}^{\infty}A^k$收敛的充要条件为$\rho(A)<1$。
\item 当$\sum_{k=0}^{\infty}A^k$收敛时,有
$$\sum_{k=0}^{\infty}A^k=(I-A)^{-1}$$
\end{enumerate}
而且存在范数$\|\cdot\|$,使得
$$\left|(I-A)^{-1}-\sum_{k=0}^{m}A^k\right|\leq \frac{\|A\|^{m+1}}{1-\|A\|}$$
\end{theorem}

\subsection{例子}
\begin{example}
(\textbf{正交回归}) 正交回归最小化点到超平面的距离.假设数据已经中心化,则回归方程无常数项,形式为:

$$\by=X\bm{\theta},\ X\in \mathcal{R}^{n\times k}$$

正交回归目标函数如下:

$$\min_{\bm{\theta}} \frac{(X\bm{\theta}-\by)^T(X\bm{\theta}-\by)}{1+\bm{\theta}^T\bm{\theta}}$$

\begin{solution}
	对目标函数求导得到
	
	$$X^T(X\bm{\theta}-\by)(1+\bm{\theta}^T\bm{\theta})-(X\theta-\by)^T(X\theta-\by)\bm{\theta}=0$$
	
	注意到$\bm{\theta}^T\bm{\theta}$和$(\by-X\bm{\theta})^T(\by-X\bm{\theta})$都是标量,所以原式实际上是两个向量相减,要求向量平行.改写如下:
	
	\begin{empheq}[left=\empheqlbrace]{align}
		\begin{bmatrix}
			X^TX& X^Ty
		\end{bmatrix}\begin{bmatrix}
		\bm{\theta}\\-1
	\end{bmatrix}&=\lambda \bm{\theta} \label{eq5a1}\\
	\begin{bmatrix}
		\bm{\theta}^T&-1
	\end{bmatrix}\begin{bmatrix}
	X^TX& X^Ty\\
	y^TX& y^Ty
\end{bmatrix}\begin{bmatrix}
\bm{\theta}\\-1
\end{bmatrix}&=\lambda 	\begin{bmatrix}
\bm{\theta}^T&-1
\end{bmatrix}\begin{bmatrix}
\bm{\theta}\\-1
\end{bmatrix}\label{eq5a2}
\end{empheq}

第一个式子的形式基本上类似于特征向量.对于第二个式子,它的一个很直观的解是


$$\begin{bmatrix}
	X^TX& X^Ty\\
	y^TX& y^Ty
\end{bmatrix}\begin{bmatrix}
	\bm{\theta}\\-1
\end{bmatrix}=\Sigma \begin{bmatrix}
\bm{\theta}\\-1
\end{bmatrix}=\lambda \begin{bmatrix}
\bm{\theta}\\-1
\end{bmatrix}$$

这就是特征值向量与特征值.同时,上式去掉矩阵的最后一行,就变成了\ref{eq5a1},所以特征向量满足原方程的解.

现在证明只有特征向量是原方程组的解.对\ref{eq5a2}改写得到

$$\begin{bmatrix}
	\bm{\theta}^T&-1
\end{bmatrix}\left(\Sigma\begin{bmatrix}
\bm{\theta}\\-1
\end{bmatrix}-\lambda \begin{bmatrix}
\bm{\theta}\\-1
\end{bmatrix}\right)=\bm{0}$$


$\bx=\begin{bmatrix}
	\bm{\theta}^T&-1
\end{bmatrix}$是$1\times(k+1)$向量,则$A\bx=0$的解集中最多包含$k+1$个独立基.而$\Sigma$的独立特征向量最多有$k+1$个,所以恰好对应起来.因此原方程组的解就是$\Sigma$的特征向量.

现在还剩下一个问题:到底是哪一个特征向量.仅由导数无法得到更多的信息,现在只能从目标函数上看.注意到目标函数实际上是

\begin{empheq}{equation}
\cfrac{\begin{bmatrix}
		\bm{\theta}^T&-1
\end{bmatrix}\Sigma \begin{bmatrix}
\bm{\theta}\\-1
\end{bmatrix}}{\begin{bmatrix}
\bm{\theta}^T&-1
\end{bmatrix} \begin{bmatrix}
\bm{\theta}\\-1
\end{bmatrix}}=\lambda
\end{empheq}

所以应该选择特征值最大的特征向量.

以上是针对没有常数项的方程,如果有常数项,则超平面应该是:

\begin{empheq}{equation}
\by=X\bm{\theta}+b
\end{empheq}

推导过程相似.

\end{solution}


\end{example}


\section{特殊矩阵运算}
\subsection{vec算子}
把矩阵按列堆积成一个向量。
\subsection{Hadamard积——Elementwise乘法}
\begin{definition}[Hadamard积]
Element-wise乘法也叫Hadamard product。
\begin{empheq}{align}
\bx\odot\by&=\by\odot\bx=\diag(\bx\by^T)=\diag(\bx)\by\\
A(\bx\odot\by)&=A\diag(\bx)\by\neq(A\bx)\odot\by\\
\sum_{i,j}(A\odot B)&=\trace(A^TB)=\trace(AB^T)\\
\bx^T(\by\odot\bz)&=\trace(\by\bx^T\diag(\bz))=\bx^T\diag(\by)\bz
\end{empheq}
\end{definition}

把矩阵$A$的每一列与一个向量$\bx$进行Element-wise乘法,相当于把行$i$乘上$\bx_i$,可以表示为
$$\diag(\bx) A$$
类似地,对列进行为:
$$A\diag(\bx)$$
记住左行右列,即左乘,是行变换。
\subsection{Kronecker积}
\subsubsection{定义}
\begin{definition}[Kronecker积]
对于任意两个矩阵$A_{m\times n},B_{p\times q}$有Kronecker积:
\begin{equation}
(A\otimes B){mp\times nq}=[A_{ij}B]
\end{equation}
也叫张量积。它满足以下性质:
\begin{enumerate}
\item 结合律:$A\otimes(B\otimes C)=(A\otimes B)\otimes C$。
\item 分配律:$(A+C)\otimes(B+D)=A\otimes B+A\otimes D+C\otimes B+C\otimes D$。
\item $(A\otimes B)(C\otimes D)=(AC)\otimes(BD)$。
\item 对共轭转置分配:$(A\otimes B)^H=A^H\otimes B^H$。
\item $\bx\in\mathbb{R}^m,\bx\in\mathbb{R}^n$,有$\bx\by^T=\bx^T\otimes \by=\by\otimes \bx^T$。
\item $\mvec(AXB)=(B^T\otimes A)\mvec(X)$
\end{enumerate}

\end{definition}

对于方阵$A\in\mathbb{R}^{m\times m},B\in\mathbb{R}^{n\times n}$的Kronecker积,还有一些特殊的性质:
\begin{enumerate}
\item 若$A,B$可逆,则$A\otimes B$可逆,且$(A\otimes B)^{-1}=A^{-1}\otimes B^{-1}$。对于Moore–Penrose伪逆\ref{moore-penrose-pseudo-inv}也是一样:$(A\otimes B)^+=A^+\otimes B^+$。
\item $\det(A\otimes B)=\det(A)^n\det(B)^m$。注意不要看错指数,指数与阶数是反的。
\item $\rank(A\otimes B)=\rank(A)\rank(B)$。
\end{enumerate}

根据Kronecker积的代数性质,有以下一些有趣的结论:
\begin{enumerate}
\item 如果$A,B$有奇异值分解$A=U_1\Lambda_1 V_1,B=U_2\Lambda_2V_2$,则$A\otimes B$有奇异值分解$(U_1\otimes U_2)(\Lambda_1\otimes \Lambda_2)(V_1\otimes V_2)$.据此可以用来导出一个定理:

记$A_{m\times m},B_{n\times n}$的特征值分别为$\lambda_i,\mu_j$,则$I_n\otimes A-B^T\otimes I_m$的特征值为$\lambda_i-\mu_j$。
\end{enumerate}

\subsubsection{应用}
Kronecker积的一大用处是用来改写线性矩阵方程组:
\begin{enumerate}
\item $AX-XB=C\implies (I_n\otimes A-B^T\otimes I_m)\mvec(X)=\mvec(C)$,对应$\mathcal{A}\bx=\bm{c}$。
\end{enumerate}
\subsection{矩阵的平方根$A^{1/2}$}
\begin{definition}[矩阵的平方根]\label{matrix-square-root}
两个方阵$A,B\in\mathbb{n\times n}$,如果$B$满足
$$A=BB$$
则称$B$为$A$的平方根。平方根不是唯一的,一般有无限多个。但恰有一个唯一的primary平方根,它的特征值的实部非负。

如果$A$是半正定矩阵,则它只有一个唯一的半正定平方根(但平方根仍然有无限多个)。这是因为半正定矩阵的特征值非负,取对角化$A=Q\Lambda Q^{-1}$,则$A^{\inv{2}}=Q\Lambda^{\inv{2}}Q^{-1}$。

如果$A$奇异,可能没有平方根。
\end{definition}
注意区别Cholesky分解$A=LL^T$,有时也用$A^{\inv{2}}=L$来表示,但这个不是标准的平方根。


\subsection{矩阵指数}
\begin{definition}[矩阵指数]
初值问题
$$\dot{\bx}=A\bx,\bx(0)=\bx_0$$
有一个唯一解
$$\bx=e^{At}\bx_0$$
其中
$$e^A=\sum_{k=0}^\infty\inv{k!}A^k$$
\end{definition}

矩阵指数具有以下性质:
\begin{enumerate}
\item $e^{\bm{0}}=I$。
\item 如果$A=\diag(\lambda_1,\cdots,\lambda_n)$,则$e^A=\diag(e^{\lambda_1},\cdots,e^{\lambda_n})$
\item 如果$XY=YX$(可交换),那么$e^Xe^Y=e^{X+Y}$。如果$X,Y$不交换,则有Lie product公式:
$$e^{X+Y}=\lim_{k\rightarrow\infty}\left(e^{\inv{n}X}e^{\inv{n}Y}\right)^{n}$$

也有 Baker–Campbell–Hausdorff公式
$$e^{X}e^Y=e^{X+Y+\inv{2}[X,Y]+\inv{12}[X,[X,Y]]+\cdots}$$
\item $e^A$一定可逆,且$(e^A)^{-1}=e^{-A}$
\item $\forall X$可逆,有$e^{X^{-1}AX}=X^{-1}e^AX$
\end{enumerate}

\subsection{skew与sym算子}
\begin{empheq}{align}
\sym(A)&=\inv{2}(A+A^T)\\
\skewop(A)&=\inv{2}(A-A^T)
\end{empheq}
\section{矩阵方程}
\subsubsection{$B+AXA^T=0$}
假设$B\in\mathbb{R}^{d\times d},A\in\mathbb{R}^{d\times k},d>k$,即$A$是一个竖矩阵。同时要求$A^TA$可逆。

\begin{empheq}{align*}
\text{原式}\implies& BA+AXA^TA=0\\
\implies& BA(A^TA)^{-1}+AX=0\\
\implies& A^TBA(A^TA)^{-1}+A^TAX=0\\
\implies&X=-(A^TA)^{-1}A^TBA(A^TA)^{-1}
\end{empheq}

\subsubsection{Riccati方程}
\paragraph*{$XCX=B$} 其中$C,B\in\mathbb{C}^{n\times n}$.这个方程在$CB$没有$R^{-}$(闭负实半轴)上的特征值时有解。其解有以下一些:
\begin{empheq}{align*}
X&=B(CB)^{-1/2}=(BC)^{-1/2}B\\
X&=C^{-1}(CB)^{1/2}=(BC)^{1/2}C^{-1}\\
X&=B^{1/2}(B^{1/2}CB^{1/2})^{-1/2}B^{1/2}
\end{empheq}
对$X,C,B\in\mathbb{R}^{n\times n}$随机生成,进行数值验证的时候发现解通常不是唯一的。这是因为平方根本身就不唯一。

\paragraph*{非对称代数Riccati方程}形如:
\begin{empheq}{align}\label{non-symemtric-riccati-eq}
XCX-XD-AX+B&=0\mtag{非对称代数Riccati方程}\\
YBY-YA-DY+C&=0\mtag{对偶方程}
\end{empheq}
其中$A\in\mathbb{R}^{m\times m},B\in\mathbb{R}^{m\times n},C\in\mathbb{R}^{n\times m},D\in\mathbb{R}^{n\times n}$,即$A,B$为方阵,而解$X\in\mathbb{R}^{m\times n}$。解可能不存在,或者有无穷多个解,实践中一般希望求解最小非负解:
\begin{enumerate}
\item $X\geq 0$
\item 如果$\tilde{X}$是另一个解,则$\tilde{X}\geq X$。
\end{enumerate}

最小非负解的存在性条件由以下给出:
\begin{definition}[非对称代数Riccati方程最小非负解的判定]
对于方程\eqref{non-symemtric-riccati-eq},如果

\begin{equation}
K=\begin{bmatrix}
D & -C\\
-B& A
\end{bmatrix}
\end{equation}
是非奇异的$M$矩阵\ref{z-m-matrix},则方程有一个最小非负解$S$,且$S$使得$D-CS$是非奇异的$M$矩阵。
\end{definition}
\section{矩阵不等式}
\subsection{涉及基本运算的不等式}
\subsubsection{$\det$}
\begin{enumerate}
\item 对任意方阵$A$,有
$$\det(A)^2\leq \prod_j \underbrace{\sum_{i} b_{ij}^2}_{\text{列元素平方和t}}$$
\item 如果$A,B$同形,则
$$\det(A^TB)^2\leq \det(A^TA)\det(B^TB)$$
\end{enumerate}
\subsection{正定与半正定矩阵}
\subsubsection{Lowner-Heinz不等式}
\begin{definition}[Lowner-Heinz不等式]
如果$A\succeq B\succeq 0,0\leq r\leq 1$,那么有
$$A^r\succeq B^r$$
\end{definition}
上式在$r>1$时一般不成立。

根据这个不等式可以导出其它一些不等式:
\begin{enumerate}
\item 如果$A\succeq B\succeq 0,r,p\geq 0,q\geq 1$且$(1+2r)q\geq p+2r$,则
\begin{empheq}{align}
(B^rA^pB^r)^{1/q}&\succeq B^{(p+2r)/q}\\
A^{(p+2r)/q}&\succeq (A^rB^pA^r)^{1/q}
\end{empheq}

还可以导出其它一些不等式:
\begin{enumerate}
\item 如果$A\succeq B\succeq 0,r\geq 0,p\geq 1$,则有
\begin{empheq}{align}
(B^rA^pB^r)^{1/p}&\succeq B^{(p+2r)/p}\\
A^{(p+2r)/p}&\succeq (A^rB^pA^r)^{1/p}
\end{empheq}
\item  如果$A\succeq B\succeq 0$,则
\begin{empheq}{align}
(BA^2B)^{1/2}&\succeq B^2\\
A^2&\succeq (AB^2A)^{1/2}
\end{empheq}
\end{enumerate}
\end{enumerate}

\subsection{范数不等式}
如果没有特别说明,所有范数都是unitarily invariant norm。
\subsubsection{矩阵乘法}
\begin{enumerate}
\item 根据范数自身的性质:
\begin{empheq}{equation}
\|AB\|\leq \|A\|\|B\|
\end{empheq}
\item $\|A\bx\|_2\leq \|A\|_2\|\bx\|_2=\lambda_{\max}(A^TA)\|\bx\|_2\leq \|A\|_F\|\bx\|_2$
\item 对于任意三个矩阵$A,B,X$和unitarily invariant norm,有
\begin{empheq}{equation}
\|AXB^T\|\leq \inv{2}\|A^TAX+XB^TB\|
\end{empheq}
\item 对于$A,B\succeq 0,1\leq 2r\leq 3, -2<t\leq 2$,有:
$$(2+t)\|A^rXB^{2-r}+A^{2-r}XB^r\|\leq 2\|A^2X+tAXB+XB^2\|$$
\item 对于同阶$A,B\succ 0,\inv{p}+\inv{q}=1$,有
$$\forall r>0,\quad \||AXB|^r\|\leq \||A^pX|^r\|^{1/p}\||XB^q|^r\|^{1/q}$$
\item 对于同阶$A,B\succeq 0$,有
\begin{empheq}{align}
\forall X,\quad \|A^{1/2}XB^{1/2}&\leq \inv{2}\|AX+XB\|\\
4\|AB\|&\leq \|(A+B)^2\|
\end{empheq}
\item 假如范数$\|\cdot\|$满足$\|I\|=1$,且$A_{n\times n}$满足$\|A\|\leq 1$,则$I-A$可逆,且
$$\|(I-A)^{-1}\|\leq \inv{1-\|A\|}$$
这可以由定理\ref{matrix-geometric-series}导出。
\end{enumerate}
\subsubsection{Hadardmon乘法}
\begin{enumerate}
\item 如果$X\succeq 0$,则
$$\forall Y,\|X\odot Y\|\leq \max_i X_{ii}\|Y\|$$
\end{enumerate}

\subsubsection{矩阵逆}

\subsubsection{矩阵指数}
\begin{enumerate}
\item $$\|e^{X+Y}-e^X\|\leq \|Y\|e^{\|X\|}e^{\|Y\|}$$
\item 记$\mu(A)=\lambda_{\max}(\inv{2}(A+A^T))$,则有
$$\|e^{At}\|_2\leq e^{\mu(t)t}$$
\end{enumerate}