\chapter{微分方程的数值算法}\label{ode-numeric}

\section{常微分方程}
\subsection{常微分方程的一般形式}
\subsubsection{主方程}
$n$阶线性常微分方程的一般形式为
\begin{empheq}{equation}
\sum_{k=1}^n a_(t,\bx)\odv[order=k]{\bx}{t}=f(t,\bx,\bx',\cdots)
\end{empheq}

线性是指为左边为导数的线性组合(系数可以是常系数,也可以是函数)。对于高阶的方程,可以通过引入额外变量的方式降阶。比如对于一元二阶线性常微分方程
$$\ddot{x}+a\dot{x}+by=f(t,x)$$
引入$x_1=x,x_2=\dot{x_1}$,则原方程可化为二元一阶线性常微分方程:
\begin{empheq}[left=\empheqlbrace]{align}
\dot{x_1}&=\dot{x}=x_2\\
\dot{x_2}&=\dot{x_1}=-ax_2-bx_1+f(t,x)
\end{empheq}

又比如非常系数的线性常微分方程也可以降阶:
$$\ddot{x}+(\sin t) \dot{x}=\cos x\implies \begin{cases}
\dot{x_1}&=\dot{x}=x_2\\
\dot{x_2}&=\dot{x_1}=-\sin t x_2+\cos x
\end{cases}$$

所以$n$阶线性常微分方程的一般形式可以表述为:
\begin{empheq}{equation}\label{1st-order-1-ode}
\dot{\bx}=\pdv{\bx}{t}=f(t,\bx) \quad \by\in\Rns
\end{empheq}
即对应一个$n$元一阶线性常微分方程组。
\subsubsection{边界条件}
\paragraph*{初值条件}给定初始时刻的值:
$$\bx(0)=\bx_0$$
对应的问题叫初值问题。初值问题适合用积分法,从初始时刻一步一步往后推。高阶方程,需要给定低阶初始导数值,比如二阶问题,要给出$\bx(0),\dot{\bx}(0)$。

初值问题适合积分法。
\paragraph*{边值问题}给定两个边界点的值
$$\bx(0)=\bx_0,\bx(T)=\bx_T$$
对应边值问题。注意,一元一阶常微分方程只有边值问题。初值问题至少需要二阶才有。

一个二阶一元方程,给定边值$x(0),x(T)$。

边值问题适合差分法。
\subsection{一元一阶线性常微分方程}
\subsubsection{积分法原理}
对于问题\eqref{1st-order-1-ode},并且假定为初值问题。按照积分的法则:
$$x_{n+1}=x_n+\int_{t_n}^{t_{n+1}} f(t,x(t))\dif t$$
右边可以用多种方法逼近。

\subsubsection{简单均值逼近}
\paragraph*{一阶时间差分}
原方程左边可以用最简单的差分逼近:
\begin{empheq}{align}
\dot{x}_n&\approx \frac{x_{n+1}-x_{n}}{h}
\end{empheq}
高阶差分格式可以参考\ref{diff-int-finite-difference}。
\paragraph*{显式Euler方法}原方程右边可以用$f(t_n,x_n)$或者$f(t_{n+1},x_{n+1})$。
就是用前一时刻的导数来逼近增量:
$$\frac{x_{n+1}-x_n}{h}=f(t_n,x_n)\implies x_{n+1}=x_n+hf(t_n,x_n)$$
可以递推计算。对应于用$t_n$处的函数值代替被积函数平均值。

\paragraph*{隐式Euler方法}用下一点处的导数逼近右边:
$$\frac{x_{n+1}-x_n}{h}=f(t_{n+1},x_{n+1})\implies x_{n+1}=x_n+hf(t_{n+1},x_{n+1})$$
显然需要解方程。性能比显式方法好。对应于用$t_{n+1}$处的函数值代替被积函数平均值。

\paragraph*{Crank-Nicolson格式}用前一时间与后一时刻的加权的方式逼近原方程右边:
\begin{empheq}{align}
&\frac{x_{n+1}-x_n}{h}=\theta f(t_{n},x_{n})+(1-\theta)f(t_{n+1},x_{n+1})\\
\implies &x_{n+1}=x_n+h(\theta f(t_{n},x_{n})+(1-\theta)f(t_{n+1},x_{n+1}))
\end{empheq}
取$\theta=\inv{2}$时,叫Crank-Nicolson格式,它是隐式方法,对应于积分的梯形法则。

\paragraph*{Predictor-Corrector校正}\label{ode-predictor-corrector}以显式Euler方法为例,先得到预测值$x_{n+1}^0=x_n+hf_n=x_n+hf(t_n,x_n)$,然后得到$f_{n+1}^0=f(t_{n+1},x_{n+1}^0)$,则根据梯形规则,预测值为
$$x_{n+1}^1=x_n+\frac{h}{2}(f_n+f_{n+1}^0)$$
这个过程可以执行多次:
\begin{empheq}{align}
x_{n+1}^0&=x_n+hf(t_n,x_n)\\
f_{n+1}^k&=f(t_{n+1},x_{n+1}^{k-1})\\
x_{n+1}^k&=x_n+\frac{h}{2}(f_n+f_{n+1}^k)
\end{empheq}
如果迭代到收敛,它可以与隐式方法具有一样的性能。以上这个过程,相当于通过不动点迭代求解
$$x_{n+1}=x_n+\frac{h}{2}(f(t_n,x_n)+f(t_{n+1},x_{n+1}))$$
相当于隐式积分法了。

\subsection{二阶一元常线性微分方程}
\subsubsection{差分法}
对于二阶一元常线性微分方程
$$a\ddot{x}+b\dot{x}=f(t,x)$$
假定为边值问题,即将时间段分成$N+1$个区间,给定$x_0,x_{N+1}$,待求解的是$[x_1,\cdots,x_N]'$,即有$N$个变量。

\paragraph*{差分格式}二阶导数取中心差分:
$$\ddot{x}=\frac{x_{t+1}-2x_t+x_{t-1}}{h^2}$$
一阶导数取前向差分。
\paragraph*{方程组}采用隐式Euler方法,则可得到$N$个方程组:
\begin{empheq}{align}
a\frac{x_{2}-2x_1+x_{0}}{h^2}+b\frac{x_1-x_0}{h}&=f(t_1,x_1)\\
a\frac{x_{3}-2x_2+x_{1}}{h^2}+b\frac{x_2-x_1}{h}&=f(t_2,x_2)\\
a\frac{x_{4}-2x_3+x_{2}}{h^2}+b\frac{x_3-x_2}{h}&=f(t_3,x_3)\\
\cdots &\cdots\\
a\frac{x_{N+1}-2x_N+x_{N-1}}{h^2}+b\frac{x_N-x_{N-1}}{h}&=f(t_{N},x_N)
\end{empheq}
这是一个非线性方程组。如果$f$与$\bx$无关,即只与$t$有关,则上面的方程组对应一个线性方程组:
\begin{empheq}{align}
\begin{bmatrix}
-2+\frac{b}{a}h & 1 & & & &  \\
1-\frac{b}{a}h & -2+\frac{b}{a}h&1 & \cdots & & \\
 & 1-\frac{b}{a}h & -2+\frac{b}{a}h&1 & & \\
&  & & \cdots& & \\
 &  & & & 1-\frac{b}{a}h&-2+\frac{b}{a}h 
\end{bmatrix}
\begin{bmatrix}
x_1\\
x_2\\
x_3\\
\cdots\\
x_N
\end{bmatrix}=\frac{h^2}{a}\begin{bmatrix}
f(t_1)\\
f(t_2)\\
f(t_3)\\
\cdots\\
f(t_N)
\end{bmatrix}+\begin{bmatrix}
1+\frac{b}{a}hx_0\\
0\\
0\\
\cdots\\
-x_{N+1}
\end{bmatrix}
\end{empheq}
对应
$$A\bx=F$$
可以看出系数矩阵为三对角矩阵,对角线上元素相同,上对角线全为1,下对角线也相同。

\subsection{其它常微分方程}
\subsubsection{多元一阶常微分方程}

\section{偏微分方程}
\subsection{差分法}

\subsection{有限元方法}
\subsubsection{弱形式}
\paragraph*{弱形式的导出}
有限元方法的起点是PDE的弱形式,也叫变分形式。考虑一个泊松方程:
$$-\Delta u(x,y)=f(x,y)$$
给两边乘上一个测试函数$v$,再在全区域上积分,即可得到弱形式:
\begin{empheq}{align}
\iint_{\Omega}(-\Delta u-f)v\dif S=\iint_{\Omega} \nabla u\cdot \nabla v - fv\dif S=0
\end{empheq}
这里用到了格林公式\ref{green-formula}。这个方程应该对所有$v$成立。说它是弱形式,是因为原方程是在每点成立,而这个方程是在全区域的积分下成立,因此强形式的解必然满足弱形式,但弱形式的解不一定满足强形式。
\subsection{有限体积法}

\subsection{同伦分析法}
