\chapter{物理学图景}
\section{物理学基本思想}
\subsection{最小作用量原理}

\subsection{守恒律}

\subsection{对称性}

\subsection{近似}

\section{物理学基本定义}
\subsection{物理量的单位}
\subsubsection{国际单位制}
\paragraph*{基本定义}
共有7个国际基本单位,其它单位可以从基本单位导出:

\begin{longtable}{ccc}
	\toprule
	含义 &   英文名   &      单位       \\ 
	\midrule
	时间 &         &    \si{\s}    \\
	长度 &         &    \si{\m}    \\
	质量 &         &   \si{\kg}    \\
	电流 &         &    \si{\A}    \\
	温度 & kelvin  &    \si{\K}    \\
	数量 &  mole   &   \si{\mol}   \\
	光强 luminous intensity& candela & \si{\candela} \\ 
	\bottomrule
\end{longtable}
\paragraph*{由基本单位导出的单位}
\begin{longtable}{cccc}
\toprule
名称 & 含义 &   英文名   &      单位       \\ 
\midrule
功 & & & \si{\J}=\si{\N\m} \\
电压 & 移动1单位电荷所做的功恰为1\si{\J}& & \si{\V}=\si{\J\per\coulomb}\\
电阻 & & & \si{\O}=\si{\V\per\A}\\
\bottomrule
\end{longtable}

\subsubsection{自然单位制}
国际单位制下的公式写起来比较繁琐,可以用自然单位制。在自然单位制下,某些物理量被标准化为1。最常见的就是光速标准化为$1$,即以$c$为单位。以下列出一些自然单位制。

\paragraph*{普朗克单位制}\label{planck-natural-unit}
\begin{empheq}{align}
c=G=\hbar=\inv{4\pi\varepsilon_0}=k_B&=1\\
e=\sqrt{\alpha}&\approx 0.08542
\end{empheq}

\subsection{物理学常数}
\begin{longtable}{cccc}
\toprule
名称 & 含义 &   英文名    &说明  \\ 
\midrule
光速 &$c=\frac{1}{\mu_0\varepsilon_0}$ & &这里包含了三个常数。\\
万有引力常数 & $G$& &\\
玻尔兹曼常数 & $k_B$& &\\
\bottomrule
\end{longtable}

\subsection{物理量的定义}
有些物理量是定义出来的,但具体的计算可能是通过一些近似得到,不同的近似会有不同的结果。

\begin{note}
\begin{enumerate}
	\item 速度可以是矢量(velocity)或者标量(speed),作为标量的速度就是矢量的$L_2$范数。比如对于圆周运动,$\bx=[r\cos\theta,r\sin\theta]^T$,则矢量速度
$$\bm{v}=\frac{\dif \bm{x}}{\dif t}=\begin{bmatrix}
	-r\sin\theta\frac{\dif \theta}{\dif t}\\
	r\cos\theta\frac{\dif \theta}{\dif t}
\end{bmatrix}$$
取绝对值就是$v=\omega r$,$\omega$为角速度。
\end{enumerate}

\end{note}
