\chapter{范畴结构}\label{category}
一个范畴由对象及对象之间的联系(态射)构成,所以它本身也属于一种结构,但又不同于一般的代数结构,范畴可以视为结构的结构,描述了不同具体结构(集合结构、群结构等等)的共性。所以本章冠以“结构”的名字。

也有人认为,集合是从元素的角度来理解世界,而范畴论是从关系的角度来理解世界。

范畴论中经常使用交换图的描述方式,交换图的含义是说,对于图中给定的起点与终点,任意一条从起点到终点的路径都是成立的,即经过不同的计算可以从起点到达相同的终点。

\section{基本定义}
\subsection{范畴}
\begin{definition}[范畴]
范畴由如下组成:
\begin{enumerate}
\item 对象$A,B,C,\cdots$组成的类$\obj(\mathcal{C})$。
\item 任意两个对象$A,B$,存在从$A$到$B$的态射$\mathcal{C}(A,B)$(也morphism,可记为$\Hom_{\mathcal{C}}(A,B)$),元素记为$f\colon A\rightarrow B$。
\item (复合)对任意三个对象$A,B,C$,存在映射$\mathcal{C}(A,B)\times \mathcal{C}(B,C)\rightarrow \mathcal{C}(A,C),(f,g)\mapsto gf$,并且满足:
\begin{enumerate}
	\item (惟一性)集合$\mathcal{C}(A_1,B_1)$与$\mathcal{C}(A_2,B_2)$相同当且仅当$A_1=A_2,B_1=B_2$。
	\item (结合律)对态射$f\colon A\rightarrow B,g\colon B\rightarrow C,h:C\rightarrow D$,有$h(fg)=(hg)f$。
	\item (单位元)对每个对象$A$,有态射$1_A\colon A\rightarrow A$,使得对任意$f\colon A\rightarrow B,g\colon C\rightarrow A$有$f1_A=f,1_Ag=g$。
\end{enumerate}
\end{enumerate}
\end{definition}

在以上的定义中,注意,“态射”可以理解为一般的“映射”。
\subsection[函子]
简单地说,函子是从一个范畴到另一个范畴的映射,通常分为共变函子与反变函子。

\begin{definition}[共变与反变函子]
给定两个范畴$\mathcal{C,D}$与范畴间的函子$F\colon \mathcal{C}\rightarrow\mathcal{D}$,如果$\mathcal{C}$中每一个对象$A$,对应$\mathcal{D}$中的$F(A)$,$\mathcal{C}$中每一个态射$f\colon A\rightarrow B$对应$F(f)\colon F(A)\rightarrow F(B)$,且
\begin{enumerate}
	\item $F(1_A)=1_{F(A)}$。
	\item $F(gf)=F(g)F(f)$。
\end{enumerate}
则称$F$为从$\mathcal{C}$到$\mathcal{D}$的共变函子。以上条件如果变成$F(f)\colon F(B)\rightarrow F(A),F(gf)=F(f)F(g)$,则称之为反变函子。
\end{definition}

从形式上看,“共变”与“反变”像是在说分配律的顺序。
\subsection{范畴的例子}

\section{高阶范畴}


\section{范畴论的应用}
