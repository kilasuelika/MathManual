\chapter{密码学}
\section{基本概念}
\subsection{密码学研究的问题}
\subsubsection{加密解密}
希望某些内容只能被一部分人知道,向其它人隐藏相关信息。是最基本的问题。其它的问题需要基于这个问题。为解决这个问题,需要构建一个系统。

对称密码体制是指双方共享密钥,整个系统由以下元素构成:

\begin{description}
\item[明文空间$P$] 所有可能的明文。
\item[密文空间$P$] 明文加密后得到密文。
\item[密钥$K$] 我们可以传送操作,比如传送加密和解密算法,但这样很困难,不同的机器运行的程序不一定相同。因此需要使用密钥的体制,密钥易于传送。
\item[加密算法$e_K$和解密算法$d_K$] $e_K: P\rightarrow C,\ d_K: C\rightarrow P$,且满足$d_K(e_K(x))=x$,但是需要注意通常$e_K(d_K(x))\neq x$,所以这里是非交换代数。
\end{description}

有对称的就有非对称的,公钥密码体制就属于这一类,除了明文和密文空间以外,现在密钥有2个:
\begin{description}
\item[公钥$SK$] 公开的,所有人都知道。
\item[私钥$PK$] 私有,只有自己知道。给定公钥生成算法和公钥$PK$,不可以得到$SK$。
\item[加密算法$E_{PK}$和解密算法$D_{SK}$] 加密$C=E_{PK}(M)$,解密$M=D_{SK}(C)$。假设想向对方传递消息,只有对方可以知道,就需要用对方的公钥加密,对方用私钥解密。
\end{description}
\subsubsection{身份验证}

\subsubsection{消息摘要}

\subsubsection{判断密码体制的安全性}
密码学中使用的都是有限域,通过穷举,一定是可以得到解的,只是穷举的效率一般很低。还可以通过随机搜索, 一个典型的例子就是“生日悖论”\ref{birthday_paradox}。



\section{加密原理}
\subsection{椭圆曲线}
椭圆曲线是由方程确定的曲线,可以是连续的,或者离散的。
\subsubsection{连续域}
由方程
$$y^2=x^3+ax+b,4a^2+27b^2\neq 0, a,b,x,y\in\mathbb{R}$$
确定的曲线再加上一个无穷远点$O$构成的集合叫椭圆曲线。在曲线上取2不相等的点,连线与曲线自身一般会交于第三个点。

在这个集合上可以定义加法$\oplus$:
\begin{enumerate}
\item $P\oplus O=O\oplus P=P$
\item 对于$P=(x_1,y_1),Q=(x_2,y_2)$:
\begin{enumerate}
	\item 如果$x_1=x_2,y_1=-y_2$,即两点的连续平行于$y$轴,那么$P+Q=O$。直线与曲线只有1个交点。
	\item 如果$P=Q$,则$P\oplus Q=2P=R'$,$R'$是过$P$的切线与曲线的第二个交点对$x$轴的对称点。即直线与曲线只有2个交点。
	\item 如果$x_1\neq x_2$,那么定义$P\oplus Q=R'$,$R'$为$P,Q$的连线与曲线自身的第三个交点,再把这个交点对$x$轴对称,或者对$y$取反。这个定义与上一个是一致的,当两个点趋近时,连线就趋于切线。

\end{enumerate}
	
详细的结果按以下公式给出。假设$y_1\neq 0,x_1\neq x_2$,那么
\begin{empheq}[left=\empheqlbrace]{align*}
	x_3&=-x_1-x_2+k^2\\
	y_3&=-y_1+k(x_1-x_3)
\end{empheq}
其中$k$是斜率:
\begin{empheq}[left={k=\empheqlbrace}]{align*}
	\frac{y_2-y_1}{x_2-x_1}&\\
	\frac{3x_1^2+a}{2y_1}&,\text{ if }P=Q
\end{empheq}

这个公式对于$P=Q,P\neq Q$都可以用。
\end{enumerate}
配上加法后,椭圆曲线构成了交换群,无穷远点是单位元。

\subsubsection{离散域}
观察到在连续域中进行计算,需要三种基本运算:加法、减法、乘法、乘法逆运算、取反。这些运算可以定义在$\mathbb{Z}_n$上。此时记椭圆曲线为$E_n(a,b)$,五种基本运算是对应的运算再$\mod n$。乘法逆运算是指$xy\equiv 1 (\mod 23)$。

给出一个实际的例子。取$E_{23}(1,1),P=(3,10),Q=(9,7)$。
\begin{enumerate}
\item 计算$-P$。结果是$(3,-10) \mod 23=(3, 13)$。
\item 计算$P+Q$。斜率是$-\fh=(-1)\times \fh=\left((-1 \mod 23) \times (2^{-1}) \right)\mod 23=11$。注意$2^{-1}=12$,因为$2\times 12 \mod 23=1$。由于$\left((a\mod n)\times b\right)\mod n=(a\times b)\mod n$,所以之前也可以用$(-1\times 12)\mod 23=11$。

现在可以按之前的公式进行计算。$x_3=-x_1-x_2+k^2=(-3-9+11^2)\mod 23=17$,$y_3=(-10+11\times (3-17))\mod 23=20$。所以$P\oplus Q=(17,20)$。
\item 计算$2P$。斜率$k=\frac{3\times 9+1}{2\times 10}=\frac{1}{4}=1\times 4^{-1}=6$。那么$x=-3-3+62=30\equiv 7(\mod 23)$,$y=-10+6\times(3-7)=12$,因此$2P=(7,12)$。
\end{enumerate}

\section{经典密码体制}
\subsection{RSA}
\subsubsection{执行过程}
\begin{enumerate}
\item 选择大素数$p,q$,通常需要512位,需要使用素数筛法来生成。
\item 计算$n=pq, \varphi(n)=(p-1)(q-1)$。
\item 选择整数$e$,满足$1\leq e\leq \varphi(n),\gcd(\varphi(n),e)=1$。
\item 计算$d\equiv e^{-1}\pmod{n}$。
\item 公开$n,e$,$e$是公钥;私有$p,q,d$,$d$是私钥。
\end{enumerate}

