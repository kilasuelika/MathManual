\subsection{Baire定理及其推论}
Baire定理是线性泛函分析的基本定理之一.以下逐步引入这个定理及其应用.
\subsubsection{Baire定理}
\paragraph*{基本Baire定理}
\begin{theorem}[Cantor交集定理]\label{cantor-intersection}
假设$X$为完备距离空间,设$(A_n)^{\infty}_{n=0}$是满足以下性质的非空闭子集序列:

\begin{enumerate}[label=(\alph*)]
	\item $A_0\supset A_1\supset\cdots$.
	\item 当$n\rightarrow \infty$时,$\diam A_n\rightarrow 0$.
\end{enumerate}

则存在唯一的$x\in X$,使得
$$\bigcap_{n=0}^\infty A_n=\{x\}$$
\end{theorem}

这个定理从形式上类似于闭区间套定理、压缩映射,利用一个逐渐缩小的集合来确定单个元素.它又类似于集合收敛.如同闭区间套定理可以刻画实数集的完备性,Cantor交集定理也可以刻画距离空间的完备性.

此外性质(b)的假设是实质性的,考虑集合列$A_n=[n, \infty)$,它满足性质(a),但不满足性质(b).

证明过程也依赖于Cauchy序列,分存在性与唯一性两个部分.

\begin{proof}
(存在性)对每个$n$,取$x_n\in A_n$,那么对于任意$m>n$,由于$A_m\subset A_n$,有
$$d(x_m,x_n)\leq \diam A_n$$

因此$\{x_n\}_0^\infty$构成Cauchy序列,设$x=\lim_{n\rightarrow \infty}x_n$.

同时$x_m\in A_n$,由于$A_n$是闭集,所以$x\in A_n$,那么$x\in \cap_{n=0}^\infty A_n$.

(唯一性)假设存在另一点$y\neq x,\ y\in \cap_{n=0}^\infty A_n$,则由于集合是逐渐缩小的,那么存在$n_0>0$,使得$\diam A_{n_0}< d(x,y)$.

但是$x,\ y$均属于交集,那么$x,\ y\in A_{n_0}$,因此$d(x,y)<\diam A_{n_0}$,与前述矛盾,所以$\cap_{n=0}^\infty A_n=\{x\}$,即集合中只包含一个元素.
\end{proof}

\begin{theorem}[Baire定理]\label{baire:proof}
假设$X$为完备距离空间,则下面两个定理等价:

\begin{enumerate}[label=(\alph*)]
\item 设$(F_n)^{\infty}_{n=0}$是$X$中的闭子集序列,且对所有$n\geq 0$,$\sint F_n=\emptyset$,则
$$\sint \left(\bigcup_{n=0}^\infty F_n\right)=\emptyset$$

\item 设$(O_n)^{\infty}_{n=0}$是$X$中的开子集序列,且对所有$n\geq 0$,$\overline{O_n}=X$,则
$$\overline{\left(\bigcap_{n=0}^\infty O_n\right)}=X$$
\end{enumerate}

\end{theorem}


直觉上说,这个定理是在讲:一组可列的闭子集,每个子集的内部是空的,则它们的交内部也是空的.可以类比:可列无限个有理数,它们合起来也不能表示一个区间,所以测度是0.

证明过程分两步;首先构造等价表述,将原问题的集合为空(就是不存在性)转换为一个集合非空的问题.既然集合是非空的,那么第二步找到这样一个元素.就证明了原问题.要点是第二步依据Cantor交集定理构造一个子集列,它们的半径趋于0.

\begin{proof}
(通过刻画集合内部为空来构造原问题的等价表述)原问题等价于证明
$$\forall \text{ 非空开子集 }O\in X,O\cap \left(X-\bigcup_{n=0}^\infty F_n\right)\neq \emptyset$$

根据集合的分配率,问题又可以转换为
$$\forall \text{ 非空开子集 }O\in X,O\cap \bigcap_{n=0}^\infty\left(X- F_n\right)\neq \emptyset$$

上式中的交集就类似于Cantor交集定理.现在只要能构造出一列子集就可以了.

(存在性)利用$F_n$是内部为空的闭子集序列,令$O_0=O$,$O$是任意一个开子集$O\subset X$.则$O_0\cap(X-F_0)\neq \emptyset$.那么
$$\exists \text{ 非空开子集 } O_1\subset X,\overline{O}_1\subset O_0\cap(X-F_0),\diam \overline{O}_1\le 1$$

比如取$O_0\cap(X-F_0)$中一个开球.由于$O_1$是开集,同样地存在
$$\exists \text{ 非空开子集 } O_2\subset X,\overline{O}_2\subset O_1\cap(X-F_1),\diam \overline{O}_2\le 1$$

类似地可以构造子集列
$$\exists \text{ 非空开子集 } O_{n+1}\subset X,\overline{O_n}\subset O_n\cap(X-F_n),\diam \overline{O}_{n+1}\le \frac{1}{n+1},n\geq 0$$

显然闭子集列$\overline{O}_n$满足Cantor交集定理,那么存在$x\in X,{x}=\cap \overline{O}_n\subset \cap (X-F_n)$.

又$x\in \overline{O}_1\subset O_0=O$,因此$x\in O$.所以
$$x\in O\cap \bigcap_{n=0}^\infty (X-F_n)$$

这就证明了转换后的问题——集合非空.

\end{proof}

证明过程中$\diam \overline{O}_n<\frac{1}{n+1}$这个约束其实是强行添加的.

从整个过程中也可以看出,最核心的就是刻画内部空与非空.
\paragraph*{基本Baire定理的等价表述}
\begin{theorem}[Baire定理的等价形式]\label{baire-eq}
设$X$是距离空间,令$F_n(n\geq 0)$是$X$的闭子集使得$X=\cup_{n=0}^\infty F_n$(实质是说集合是可列个子集的并),那么有以下结论:
\begin{enumerate}[label=(\alph*)]
\item\label{baire-eq-a1} 如果$\forall n\geq 0,\sint F_n=\emptyset$,那么$X$是不完备的。
\item 如果$X$是完备的,那么$\exists n_0\geq 0,\sint F_{n_0}\neq\emptyset$。即内部非空。
\end{enumerate}
\end{theorem}
本定理的一个简单推论就是二维平面不能表示成可列条直线的并集,因为直线的内部是空的。

另外由\ref{baire-eq-a1}可以得到以下结论:
\begin{theorem}[]
无穷维Banach空间不可能具有可列无穷Hamel基。

特别地,由单变量或者多变量组成的多项式空间,不可能装备范数,成为Banach空间(所有$n$次多项式是可列、无穷的,而且构成Hamel基)。
\end{theorem}
证明的主要思路是证明集族
$$F_n=\Span \{e_j\}_{j=0}^n$$
有
$$\int F_n=\emptyset$$
用反证法。

那么如何理解多项式空间中的不完备性呢?

其一是可以构造收敛点不在多项式空间中的Cauchy列。级数展开就是一个例子。比如对于$e^x$,它的级数展开在任意一点都是收敛的,但显然$e^x$不属于多项式空间。

\subsubsection{Baire定理的推论}
以下定理中Banach-Steinhaus定理、Banach开映射定理、Banach闭图像定理是线性泛函分析的三大基石。
\paragraph*{Banach-Steinhaus定理}这个定理是说,由值域的有界可以导出线性算子范数的有界;或者由point-wise有界导出normed有界。也叫一致有界原理。
\begin{theorem}[Banach-Steinhaus定理]\label{Banach-steinhaus}
设$X$是Banach空间,$Y$为赋范向量空间,而$(A_i)_{i\in I}$为映射$A_i\in \mathcal{L}(X;Y)$构成的算子族,满足
$$\forall x\in X,\sup_{x\in \mathcal{I}}\|A_ix\|_Y<\infty$$
而
$$\sup_{i\in \mathcal{I}}\|A_i\|_{\mathcal{L}(X;Y)}<\infty$$
\end{theorem}
证明分为两个部分,首先构造一个集族,利用Baire定理证明存在一个闭子集,然后对$x$重新进行表示,利用三解不等式来得到算子范数有界。
\begin{proof}
\circled{1}首先证明存在闭子集。取集合
$$F_n=\{x\in X\mid \sup_{i\in\mathcal{I}}\|A_ix\|\leq n\}$$
这里的$i$是遍历所有算子。对于任意$x\in X$,由假设$\sup_{x\in \mathcal{I}}\|A_ix\|_Y<\infty$可知,存在整数$n(x)\geq 0$,有$\sup_{x\in \mathcal{I}}\|A_ix\|_Y<n(x)$,那么$x\in F_{n(x)}$。所以
$$X=\bigcup_{n=0}^\infty F_n$$
原因是$x$是任取的,任意取它都在$F_{n(x)}$中,而遍历$n(x)$相当于$n$,所以$X$可以由$F_n$取并得到。

$F_n$的另一个等价表述是
$$F_n=\bigcap_{i\in\mathcal{I}}\{x\in X\mid \|A_ix\|\leq n\}$$
这里是取交集,因为原始定义中是上界小于$n$,就是所有算子都必须满足小于$n$,就是取交。

由这个定义可知$F_n$是$X$中闭子集的交集,于是它也是闭集。

根据$X$的完备性,可以使用Baire定理,因此
$$\exists n_0\geq 0,\sint F_{n_0}\neq \emptyset$$
因此$\exists (x_0\in F_{n_0},r>0),\overline{B(x_0;r)}\subset F_{n_0}$。由$F_{n_0}$的定义,有
$$\forall z\in\overline{B(x_0;r)},i\in\mathcal{I}, \|A_iz\|\leq n_0$$

\circled{2}重新表示$x$,取$z=\left(x_0+r\frac{x}{\|x\|}\right)\subset \overline{B(x_0;r)}$,那么
$$x=\frac{\|x\|}{r}(z-x_0)$$
于是
\begin{empheq}{align*}
\|A_ix\|&\leq \frac{\|x\|}{r}(\|A_iz\|+\|A_ix_0\|)\\
&\leq \inv{r}(n_0+\|A_ix_0\|)\|x\|\\
&\leq \inv{r}\left(n_0+\sup{i\in\mathcal{I}}\|A_ix_0\|\right)\|x\|,\forall i\in \mathcal{I},x\in X
\end{empheq}
由假设$\sup{i\in\mathcal{I}}\|A_ix_0\|<\infty$,于是
$$\sup_{i\in\mathcal{I}} \|A_i\|=\sup_{x\in X}\frac{\|A_ix\|}{\|x\|}\leq \inv{r}\left(n_0+\sup{i\in\mathcal{I}}\|A_ix_0\|\right)<\infty$$
\end{proof}
在上面的证明中,第\circled{1}步是不可以省略的,假如只有第\circled{2}步,那么$z$可能压根不属于$X$(比如取一组离散的点),那么也就没法使用三角不等式了。

另一个初等证明由\cite{Alan_D_Sokal_2011}给出,不依赖于Baire定理,它使用了反证法,首先假设$\sup_{i\in \mathcal{I}}\|A_i\|=\infty$,然后导出$\exists x,A_ix\rightarrow\infty$。

\paragraph*{Banach开映射定理}一个Banach空间到另一个Banach空间中的连续线性线性算子是开映射。
\begin{theorem}[Banach开映射定理]
设$X, Y$是Banach空间,$A\in\mathcal{L}(X;Y)$是满射。

那么$X$的任意开子集在映射$A$下的直接像$A(U)$是$Y$的开子集。
\end{theorem}
所谓开集就是每个点都有一个开领域,具体地说就是给定任意开集$U\in X$,对于$\forall y\in A(U)$,希望找一个开球,证明主要从这个目标入手。

\paragraph*{Banach开映射定理的推论}双射的逆也是连续线性算子。
\begin{theorem}[Banach开映射定理]
设$X, Y$是Banach空间,$A\in\mathcal{L}(X;Y)$是满双射。则$A^{-1}\in\mathcal{L}(X;Y)$。
\end{theorem}
