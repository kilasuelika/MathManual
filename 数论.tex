\chapter{数论}
\section{基本概念}
\subsection{素数}

\subsection{模运算}
有人认为加、减、乘、除、模是五种最基本的运算,用它们可以表示其它的运算,同时,只有定义了这5种运算,才是完备的。模运算通常用在离散数学中,所以一般的科学技术用的并不多。它可以用来表示周期性。
\subsubsection{同余关系}
同余关系是指
$$a\equiv b \pmod n$$
利用模$n$同余关系可以将整数划分为$n$等价类,这些等价类构成模$n$的完全剩余系,记为
$$\mathbb{Z}_n=\{0,1,\cdots,n-1\}$$

从完全剩余系中去掉与$n$不互素($\gcd(n,k)=1$)的那些数(显然0不是不互素的,$\gcd(n,0)\neq 1$),剩下的叫简化剩余系,比如
$$Z_9^\star=\{1,2,4,5,7,8\}$$
注意$3,6$与9是不互素的。
\subsubsection{模运算的性质}
\paragraph*{四则运算}以同余关系定义的运算法则有以下一些。

如果$a\equiv b\pmod n$,那么有
\begin{empheq}{align*}
a+k\equiv b+k&\pmod n\\
a^k\equiv b^k&\pmod n
\end{empheq}

以同余关系定义,有点类似于对方程两边进行操作,从一个预定义的关系出发。也可以以直接代数运算来定义,此时是从一边进行推导到另一边。

\begin{empheq}{align*}
(a\bmod n+b\bmod n)\bmod n&=(a+b)\bmod n\\
(a\bmod n\times b\bmod n)\bmod n&=(a\times b)\bmod n
\end{empheq}

\paragraph*{乘法逆}假设$gcd(n,k)=1$,那么存在$k^{-1}$,有$k\times k^{-1}\equiv 1  \pmod n$。这里是对互素的情况,如果不互素,一般好像定义乘法逆为0。

\paragraph*{重复序列}使用模运算$x \mod n$可以生成$0,1,\cdots,n-1,0,1,\cdots,n-1,\cdots$的重复序列。使用$((x-1)\mod n) +1$可以生成$1,\cdots,n,1,\cdots,n,\cdots$的重复序列,在数组循环的时候很有用。
\subsection{Euler定理}
定义欧拉函数$\varphi(n)$,其含义是不超过$n$且与$n$互素的元素个数,也相当于简化剩余系的个数。
\begin{empheq}{align*}
\varphi(1)&=1\\
\varphi(p)&=p-1,\ p\text{为素数}\\
\varphi(p^k)&=p^{k-1}(p-1)=p^k-p^{k-1}=p^k\left(1-\inv{p}\right),\ \text{如果}p\text{为素数}\\
\varphi(n_1n_2)&=\varphi(n_1)\varphi(n_2),\ \text{如果}\gcd(n_1,n_2)=1\\
\varphi(n)&=n\prod_{i=1}^{n}\left(1-\inv{p_i}\right),\ \text{如果}n=\prod_{i=1}^{n}p_i^{a_i},p_i\text{为素数}
\end{empheq}