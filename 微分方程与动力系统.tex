\chapter{微分方程与动力系统}\label{ode-dynamic-system}
\section{常微分方程}
\subsection{常微分方程的求解}
\subsubsection{一览表}
\begin{longtable}{cl}
\toprule
\bf{方程形式}& \hfill\bf{解}\\
\midrule
$y'(x)+p(x)y=q(x)$& $y=e^{-\int p(x)\dif x}\left(\int q(x)e^{-\int p(x)\dif x}+C\right)$\\
\midrule
\multirow{5}{*}{$ay''+by'+cy=0$}&首先求解特征方程:$ar^2+br+c=0$,得到两个解$r_1,r_2$:\\
&\circled{1} $r_1,r_2$为两个不等实根,则$y=C_1e^{r_1x}+C_2e^{r_2x}$\\
&\circled{2} $r_1=r_2=r$,为重根,则$y=C_1e^{rx}+C_2xe^{rx}$\\
&\circled{3} $r_1,r_2$为两个纯虚根$\pm i\omega$,则$y=C_1\cos(\omega x)+C_2\sin(\omega x)$\\
&\circled{4} $r_1,r_2$为两个复根$r\pm i\omega $,则$y=e^{rx}(C_1\cos(\omega x)+C_2\sin(\omega x))$\\
\bottomrule
\end{longtable}

\subsection{Lyapunov稳定性理论}\label{ode-lyapunov}
Lyapunov理论主要用来考查常微分方程
\begin{equation}\label{lyapunov-ode}
\dot{\bx}=\bm{f}(t,\bx),\bx(t_ 0)=\bx_0
\end{equation}
的稳定性。

首先要定义什么是稳定性。

\subsubsection{Lyapunov稳定性}
\begin{theorem}[Lyapunov稳定性]
\begin{description}
\item[平衡点] $\bx^*$称为平衡点,如果$\forall t\geq 0, f(\bx^*,t)=0$。
\item[稳定] 一个平衡点$\bx_e$称为稳定的,如果$\forall t_0\geq 0,\epsilon>0$,存在标量$\delta(t_0,\epsilon)$满足
$$\|\bx_0-\bx_e\|<\delta \implies \forall t>t_0,\|\bx_t-\bx_e\|<\epsilon$$
含义是说如果初始点接近平衡点,则轨迹始终停留在距离平衡点的一个小区域内。
\item[一致稳定] 在稳定的定义中,如果$\delta$可以与$t_0$无关,则称为一致稳定。
\item[渐近稳定] $\bx_e$稳定且为吸引子。\textbf{吸引子}定义为:$\forall t_0\geq 0$,存在$\delta(t_0)$满足
$$\|\bx_0-\bx_e\|<\delta \implies \lim_{t\rightarrow \infty}\|\bx_t-\bx_e\|=0$$
吸引子的含义是说如果初始点距离平衡点够近,则最终轨迹收敛于平衡点。
\item[一致渐近稳定] $\bx_e$一致稳定且轨迹一致收敛于$\bx_e$。后者定义为存在$\delta>0$和一个函数$\mathbb{R}_+\times \mathbb{R}^n\rightarrow \mathbb{R}_+$,$\gamma$满足$\forall \bx_0\in B_{\delta},\lim_{\tau\rightarrow \infty}\gamma(\tau,\bx_0)=0$。然后有
$$\|\bx_0-\bx_e\|<\delta \implies \|\bx_t-\bx_e\|\leq \gamma(t-t_0,\bx_0)$$
这里与一致稳定加上渐近稳定非常像,只是在渐近的定义上有所不同,渐近稳定中考虑的是极限,而一致渐近稳定考虑的是距离的上界,$\gamma$刻画了收敛速率。
\item[全局渐近稳定] $\bx_e$稳定且$\forall \bx_0\in\Rns,\lim_{t\rightarrow\infty} \bx_t=\bx_e$.含义为对于\emph{任意}初始值,轨迹最终趋于$\bx_e$。
\item[指数稳定] $\bx_e$稳定,且存在$m,\alpha>0$,满足
$$\forall \bx_0\in B_h,t\geq t_0\geq 0,\|\bx_t-\bx_e\|\leq me^{-\alpha(t-t_0)}\|\bx_0\|$$
$B_h$是$\bx_e$的邻域,$\alpha$称为收敛速率。含义是轨迹与均衡点的距离按指数衰减。

大致可以看出,指数稳定是渐近稳定的特例。
\end{description}
\end{theorem}
以上的定义中上,核心点主要有三个:稳定、吸引子、指数稳定。稳定与吸引子容易混淆。前者指的是轨迹始终停留在一个小区域内,而后者说的是极限情况。吸引不一定稳定,因为在中途,轨迹可能脱离这个小区域,也就是说无论初始距离取得多么近,在收敛之前,它会脱离小区域。

稳定也不一定吸引,可能在平衡点附近一直波动,但就是不收敛。
\subsubsection{能量函数}
\begin{definition}[能量函数与正定函数]
一个连续、严格\emph{单调递增}函数,且满足$\alpha(0)=0$的函数$\alpha:\mathbb{R}^+\rightarrow \mathbb{R}^+$称为{\heiti K族函数}(隐含$\alpha(\cdot)>0$。如果它还满足当$p\rightarrow \infty$时,有$\alpha(p)\rightarrow \infty$,则称之为{\heiti KR函数}。

一个连续函数$v(\bx,t): \mathbb{R}^n\times \mathbb{R}_+\rightarrow \mathbb{R}_+$称为{\heiti 局部正定函数}(l.p.d.f),如果它满足,对于一些$h>0$和一些$K$函数$\alpha$,有:
$$v(0,t)=0,\text{且} \forall \bx\in B_h,t>0, v(\bx,t)\geq \alpha(|\bx|)$$

一个l.p.d.f局部地像一个能量函数。而一个{\heiti 正定函数}全局地像{\heiti 能量函数}。

具体地说,一个连续函数$v(\bx,t): \mathbb{R}^n\times \mathbb{R}_+\rightarrow \mathbb{R}_+$称为 正定函数(p.d.f),如果它满足,对于一些KR函数$\alpha$,有
$$v(0,t)=0,\text{且}\forall \bx\in\mathbb{R}^n,t\geq 0, \text{有}v(\bx,t)\geq \alpha(|\bx|)$$

同时还有{\heiti 减函数}。一个连续函数$v(\bx,t): \mathbb{R}^n\times \mathbb{R}_+\rightarrow \mathbb{R}_+$称为减函数(decrescent函数)。如果存在一些K函数$\beta$,满足
$$\forall \bx\in B_h,t\geq 0,v(\bx,t)\leq\beta(|\bx|)$$
\end{definition}
局部正定函数相当于在局部,函数的下界为一个单调递增函数(准确地说是K函数),或大于一个$K$函数。而正定函数是全局性的,含义为函数全局地大于一个KR函数。减函数是说函数小于一个K函数。

局部正定函数与正定函数的区别在于下界的性质和$\bx$的范围。

\begin{example}
以下给出一些能量函数的例子:
\begin{enumerate}
\item $|\bx|^2$:减p.d.f.
\item $\bx^TP\bx$,$P$正定矩阵:减p.d.f.。$\lambda_{\min} |\bx|^2\leq  \bx^TP\bx\leq \lambda_{\max}|\bx|^2$。
\item $(t+1)|\bx|^2$:p.d.f.$(t+1)|\bx|^2\geq |\bx|^2$。
\item $e^{-t}|\bx|^2$:减函数。上界可以取$e^{-t_0}|\bx|^2$.
\item $\sin^2(|\bx|^2)$:减l.p.d.f.
\item $e^t\bx^TP\bx$,$P$不正定:以上全不是。
\item $v(\bx,t)$与$t$无关,则它为减函数。
\end{enumerate}
\end{example}


\subsubsection{基本Lyapunov定理}
\begin{theorem}[基本Lyapunov定理]
对于任意能量函数,以下给出了判断ODE\eqref{lyapunov-ode}稳定性的条件:
\begin{longtable}{|c|c|c|}
\hline
$v$ & $-\dot{v}$ & \textbf{结论}\\
\hline
l.p.d.f. & 局部$\geq 0$ &稳定\\
减l.p.d.f. & 局部$\geq 0$ & 一致稳定\\
减l.p.d.f. & l.p.d.f. & 一致渐近稳定\\
减p.d.f. & p.d.f. &全局一致渐近稳定\\
\hline 
\end{longtable}

其中
$$\dot{v}=\sum_{i=1}^n\pdv{v}{x_i}f_i+\pdv{v}{t}$$

需要强调,以上给出的是一个充分而非必要条件,如果一个Lyapunov函数下,以上条件不满足,只能说在这个函数下不能给出稳定性;但在其它函数下仍然可能稳定。
\end{theorem}

\subsubsection{不稳定性}
\paragraph*{自治系统}以下几个定理涉及$f$与时间无关的ODE:
\begin{equation}
\dot{x}=f(\bx),\bx(t_0)=\bx_0
\end{equation}

\begin{theorem}[Lyapunov不稳定性定理]
记$\bx_e$为一平衡点。如果存在一个函数$V:\mathbb{R}^n\rightarrow \mathbb{R}$,满足:
\begin{enumerate}
\item $\forall \bx\in\{\bx:V(\bx)\leq 0\}, \dot{V(\bx)}\leq 0$.注意这里$\bx$是$t$的函数。
\item $V(x_0)< V(\bx_e)$
\end{enumerate}
则轨迹$\bx(t)$不会收敛于$\bx_e$。
\end{theorem}

以上定理其实比较简单,因为初始的能量小于$V(\bx_e)$,而$\dot{V}$又是小于等于0的,所以随着时间的发展,能量会起来越小,且小于$V(\bx_e)$,自然不会收敛于$\bx_e$。

\begin{theorem}[Lyapunov发散定理]
记$\bx_e$为一平衡点。如果存在一个函数$V:\mathbb{R}^n\rightarrow \mathbb{R}$,满足:
\begin{enumerate}
\item $\forall \bx\in\{\bx:V(\bx)< 0\}, \dot{V(\bx)}< 0$
\item $V(x_0)< 0$
\end{enumerate}
则轨迹$\bx(t)$无界。
\end{theorem}

这个定理也比较简单。首先必然有$V(\bx(t))<V(\bx_0)$,这与$\dot{V(\bx)}<0$和初始值 $V(x_0)< 0$有关。现在假设轨迹有界,即$\|\bx(t)\|\leq R$,那么集合$\{\bx:V(\bx)\leq V(\bx_0),\|\bx\|\leq R\}$为紧致集合,则有正数$\beta$满足$-\dot{V}\leq -\beta$,于是$V(\bx(t))\leq V(\bx_0)-\beta t$,那么$V(\bx(t))\rightarrow \infty$,因此无界。


\subsubsection{分析案例}
\paragraph*{塌缩函数}此处是指形如$\bm{f}(\bx,t)=\bm{g}(s(\bx),t)$,即向量$\bx$最终可以归结为一个标量进入模型。

\begin{example}
什么
\end{example}

\paragraph*{非塌缩函数}

\begin{example}
\begin{empheq}[left=\empheqlbrace]{align*}
\dot{x_1}&=x_2\\
\dot{x_2}&=-x_2-(2+\sin t)x_1
\end{empheq}
显然$\bm{0}$为平衡点。取能量函数
$$v(\bx,t)=x_1^2+\frac{x_2^2}{2+\sin t}$$
由于
$$x_1^2+x_2^2=|\bx|^2\geq v(\bx,t)\geq x_1^2+x_2^2/3\geq 1/3|\bx|^2$$
因此它是减p.d.f.。

再考虑$-\dot{v}$:
$$-\dot{v}=\frac{x_2^2(4+2\sin t+\cos t)}{(2+\sin t)^2}\geq 0$$
因此是一致稳定的。但不是一致渐近稳定,因为$-\dot{v}$只与$\bx_2$有关,找不到一个函数K函数$\alpha(|\bx|)$,使得$-\dot{v}$的下界为它。
\end{example}

\begin{example}
{\heiti 极限环系统}

\begin{empheq}[left=\empheqlbrace]{align*}
\dot{x_1}&=-x_2+x_1(x_1^2+x_2^2-1)\\
\dot{x_2}&=x_1+x_2(x_1^2+x_2^2-1)
\end{empheq}
显然$\bm{0}$为平衡点。取能量函数
$$v(\bx,t)=x_1^2+x_2^2$$
它是减p.d.f.。

再考虑$-\dot{v}$:
$$-\dot{v}=-2(x_1^2+x_2^2)(x_1^2+x_2^2-1)=-2|\bx|^2(|\bx|^2-1)$$
在$|\bx|<1$的范围内,$-\dot{v}$相对于$|\bx|$单调递增;在较大的范围内,$-\dot{v}<0$,因此只是l.p.d.f.。所以$\bm{0}$是一致渐近稳定的,但不是全局一致渐近稳定。
\end{example}
\subsection{其它稳定性理论}
\subsubsection{线性方程与线性近似的稳定性}
\begin{theorem}[线性ODE方程的稳定性]
一个线性方程
$$\dot{\bx}=A\bx$$
则它的稳定性与$A$的特征值密切相关:
\begin{enumerate}
\item $\bx=0$渐近稳定$\iff \forall \lambda_i(A),\Re(\lambda_i)<0$,即特征值实部全为负。
\item $\bx=0$稳定$\iff \Re(\lambda_i)\leq 0$,且实部为负的特征根对应的Jordan块为一阶。
\item $\bx=0$不稳定$\iff \exists \lambda_i, \Re(\lambda_i)>0$
\end{enumerate}
\end{theorem}

线性方程的稳定性与线性近似的稳定性密切相关:

\begin{theorem}[线性近似的稳定性]
如果$\bm{f}$是非线性的,则在平衡点附近进行线性化:
$$\dot{\bx}=A(t)\bx+\bm{N}(t,\bx)$$

$\bm{N}$满足
\begin{enumerate}
\item $N(t_0, \bx_e)=0$
\item $\forall t\geq t_0,\lim_{|\bx|\rightarrow|\bx_e|}\frac{|N(t,\bx)|}{|\bx|}=0$
\end{enumerate}

如果$A(t)=A$为一常矩阵,则原来的非线性方程的稳定性与线性化后的方程的稳定性相关:
\begin{enumerate}
\item $\bx=0$渐近稳定$\iff \forall \lambda_i(A),\Re(\lambda_i)<0$,即特征值实部全为负。
\item $\bx=0$不稳定$\iff \exists \lambda_i, \Re(\lambda_i)>0$
\end{enumerate}

需要注意,线性近似法相比纯线性方程的稳定性的区别在于:线性挖只适用于\emph{渐近稳定}与\emph{不稳定}的情形,不能判别稳定。

如果$A(t)$不为常矩阵,则没有以上结论。
\end{theorem}




\subsection{不稳定性理论}
\subsubsection{爆破}

\subsubsection{刚性}

\subsubsection{极限环}

\section{偏微分方程}

\subsection{概论}
\subsubsection{PDE的分类理论}
偏微分方程就是指含有多个自变量的导数的方程。

\paragraph*{一阶PDE}只含有一阶导数的PDE,形如
$$\sum a_i(\bx)\pdv{u(\bx)}{x_i}=F(u)$$
的称为一阶线性PDE。如果$a_i$与$u$有关,形如:
$$\sum a_i(\bx,u(\bx))\pdv{u(\bx)}{x_i}=F(u)$$
的称为一阶拟线性PDE。不是线性、拟线性的叫一阶非线性PDE。

\paragraph*{二阶PDE}最高含有二阶导数的PDE为线性PDE。$n$个自变量(时间$t$也算其中的一个)的二阶线性PDE的一般形式为:
\begin{empheq}{equation}
\sum_{i=1}^n\sum_{i=1}^na_{ij}\pdv[order=2]{u}{x,y}+\sum_{i=1}^nb_i\pdv{u}{x_i}+cu(\bx)=f
\end{empheq}
系数$a_{ij}$构成系数矩阵$A=[a_{ij}]$。依据系数矩阵在某点$\bx$处的特征值分布:
\begin{description}
\item[椭圆型] $A$的特征值同号。
\item[双曲型] $A$的特征值有$n-1$个同号,与剩下的一个异号。
\item[抛物型] 特征值有一个为0,其它同号。
\end{description}
称PDE在该点处为对应的型。如果PDE在整个区域内都为此型,即称为对应的PDE型;如果在某些部分为一型,另一些部分为其它型,称为混合型PDE。
\paragraph*{例子}以下给出一些PDE的例子与其对应的分类:
\begin{longtable}{ccc}
\toprule
名称 & 形式 & 说明\\
\midrule
拉普拉斯方程& $\Delta u=f$ & 二阶椭圆。\\
波动方程& $\pdv[order=2]{u}{t}-c^2\pdv[order=2]{u}{x}=0$ &二阶双曲。系数矩阵的特征值为$1,-c^2$。\\
热传导& $\pdv{u}{t}=k\pdv[order=2]{u}{x}$& 二阶抛物。\\
\bottomrule
\end{longtable}

\subsection{计算方法}


\subsection{分析方法}


\section{离散时间动力系统}
\subsection{表示论}
\subsubsection{常系数线性系统与矩阵乘法}
给定一个线性系统
$$y_{n+2}=2y_{n+1}+y_{n}$$
可以表示为矩阵乘法
\[
\begin{bmatrix}
y_{n+3}\\
y_{n+2}
\end{bmatrix}=\begin{bmatrix}
5 & 2\\
2 & 1
\end{bmatrix}\begin{bmatrix}
y_{n+1}\\
y_{n}
\end{bmatrix}\]

比如给定初始值$[y_0,y_1]=[1,1]$,那么直接迭代计算$y_4=17,y_5=41$,也可以用两次矩阵计算
\[\begin{bmatrix}
	y_{5}\\
	y_{4}
\end{bmatrix}=\begin{bmatrix}
	5 & 2\\
	2 & 1
\end{bmatrix}^2\begin{bmatrix}
	y_{1}\\
	y_{0}
\end{bmatrix}=\begin{bmatrix}
41\\17
\end{bmatrix}\]
利用矩阵乘法,一次可以计算2个值。显然原本的迭代增长速度是指数的,所以矩阵连乘可以表达指数增长。而且可以利用特征值分解来计算矩阵乘法。

那么一个显然的问题是,矩阵连乘能不能表示多项式增长?当然是可以的。

考虑矩阵
\[M=\begin{bmatrix}
	1 & 2\\
	0 & 1
\end{bmatrix}\]
可以算出
\[M^n=\begin{bmatrix}
	1 & 2n\\
	0 & 1
\end{bmatrix}\]
这是多项式增长的。而且注意到,其实$M$的特征值是2个1。

其实对于矩阵
\[
M=\begin{bmatrix}
1& a_{12} & a_{13}\\
0& 1 & a_{23}\\
0&0&1
\end{bmatrix},M^n=\begin{bmatrix}
1& na_{12} & \frac{n(n+1)}{2}a_{13}a_{23}+na_{13}\\
0& 1 & na_{23}\\
0&0&1
\end{bmatrix}
\]

可以看出是多项式增长的。一个简单的想法就是,对于三角阵,如果对角线上都是1,那就是多项式增长,这也部分说明了为什么对角线元素比较重要。

\subsubsection{使用矩阵或纯加法计算多项式}
一个多项式可以写成 $a_nx_1^n=a_n(x_0+\dif x)^n=a_n\sum_{k=0}^{n }\alpha_k x^k\dif^{n-k}x$的形式,其中$\alpha_k$是多项式系数。比如,

$$
\begin{bmatrix}
	a_4x_1^4\\
	a_4x_1^3\dif x\\
	a_4x_1^2\Dif2 x\\
	a_4x_1\Dif3 x\\
	\Dif4 x
\end{bmatrix}=\begin{bmatrix}
	\binom{4}{4} & \binom{4}{3} & \binom{4}{2} & \binom{4}{1} &\binom{4}{0}\\
	0& \binom{3}{3} & \binom{3}{2} &\binom{3}{1} &\binom{3}{0} \\
	0& 0& \binom{2}{2} & \binom{2}{1} & \binom{2}{0}\\
	0& 0&0& \binom{1}{1}&\binom{1}{0}\\
	0& 0& 0& 0& 1
\end{bmatrix}\begin{bmatrix}
	a_4x_0^4\\
	a_4x_0^3\dif x\\
	a_4x_0^2\Dif2 x\\
	a_4x_0\Dif3 x\\
	\Dif4 x
\end{bmatrix}=\begin{bmatrix}
	1 & 4 & 6 & 4 &1\\
	0& 1 & 3 &3 &1 \\
	0& 0& 1 & 2 & 1\\
	0& 0&0& 1&1\\
	0& 0& 0& 0& 1
\end{bmatrix}\begin{bmatrix}
	a_4x_0^4\\
	a_4x_0^3\dif x\\
	a_4x_0^2\Dif2 x\\
	a_4x_0\Dif3 x\\
	\Dif4 x
\end{bmatrix}
$$

通过矩阵连乘,我们可以计算 $x_i^n,i\geq 1$。不过这里还有标量乘法。假如只用加法,\circled{1} 希望矩阵的元素属于 $\{-1,0,1\}$,或者  \circled{2} 直接用加法代替乘法,需要 $2^4+2^3+2^2+2^1+2^0-\frac{(1+5)5}{2}$ 次加法来代替矩阵乘法。后者显然是不合适的。

现在来考虑一个更抽象的问题,计算:
$$\bm{y}_{n+1}=L\bm{y}_n \text{ ,且 } L \text{是上三角阵且} L_{ii}=1$$

考虑方法\circled{1}, 也就是通过基变换的方式。希望找到矩阵 $M,A$使得
$$A\by_{n+1}=MA\by_{n} \text{ 且 }M_{ij}\in\{-1,0,1\}$$
这样一来 $A\by_{n+1}=AL\by_n=MA\by_n$。一个简单的条件是$AL=MA$。注意上面这个变换的含义,$A$是变换矩阵,我们同时对$\by_{n+1},\by_n$进行变换,使得$A\by_{n+1}$恰好可以通过$A\by_n$进行行的加减法得到。

显然这个问题是不定的,因为参数过多。我们可以随便选择$M$ 然后求解$AL=MA$ 得到$A$ ,具体来说是求解$Px=0$ ,其中$x$ 是将 $A$中元素堆叠形成的向量。

在我们的问题中,需要计算 $a_4x^4$ ,它是$\by_n$的第一行。那么$A$的第一行应该是$[1,\bm{0}]$ ,也就是说我们寻找 的变换不改变 $\by_n$的第一行。并且 $M$ 必须是上三角阵,因为每一行为线性组合的系数,而高次项的线性组合不可能得到低次项。于是$\forall i<j, M_{ij}=0$。即使加上这些约束,问题还是不定的。 

现在我们考虑用尽可能少的加法,那么至少需要的加法次数是1。于是 $M$的形式应该像:
\[M=\begin{bmatrix}
	1 & 1 &  &  &  \\
	& 1& 1& &\\
	& & 1& 1& \\
	& & & 1& 1\\
	& & & & 1
\end{bmatrix}\]

由于$A$的第一行固定为$[1,\bm{0}]$,我们可以按顺序求解出 $A$的每一行。比如对于第二行可以构造方程:
$$\begin{aligned}
	\underbrace{\begin{bmatrix}
			1 & 0 & 0 & 0& 0\\
			0 & x_2 & x_3 & x_4 &x_5\\
	\end{bmatrix}}_{A}\underbrace{\begin{bmatrix}
			1 & 4 & 6 & 4 &1\\
			0& 1 & 3 &3 &1 \\
			0& 0& 1 & 2 & 1\\
			0& 0&0& 1&1\\
			0& 0& 0& 0& 1
	\end{bmatrix}}_{L}&=\underbrace{\begin{bmatrix}
			1 & 4 & 6 & 4 &1\\
			0& x_2& 3x_2+x_3&3x_2+2x_3+x_4&x_2+x_3+x_4+x_5
	\end{bmatrix}}_{P}\\
	=\underbrace{\begin{bmatrix}
			1 & 1 &  &  &  \\
			& 1& 1& &\\
			& & 1& 1& \\
			& & & 1& 1\\
			& & & & 1
	\end{bmatrix}}_{M}\underbrace{\begin{bmatrix}
			1 & 0 & 0 & 0& 0\\
			0& x_2 & x_3 & x_4 &x_5\\
	\end{bmatrix}}_{A}&=\underbrace{\begin{bmatrix}
			1 & x_2 & x_3 & x_4 &x_5\\
			& & \cdots & & 
	\end{bmatrix}}_{Q}
\end{aligned}
$$

对比$P$与$Q$,有$[1, x_2,x_3,x_4,x_5]=[1,4, 6,4,1]$。它恰好是$L$的第一行。那么$P$的第二行已经固定了。现在可以用$Q$的第2行求解$A$的第3行$\cdots$最终解出$A$的全部行。 用数学的语言来表示这个迭代过程:
\[\begin{aligned}
	P_{i,}&=A_{i,}L\\
	Q_{i,}&=A_{i,}+A_{i+1,}\\
	P_{i,}&=Q_{i,}
\\
	s.t.&\ A_{0,},L\text{ 已知}
\end{aligned}\]
其中 $A_{i,}$ 表示$A$的第$i$行。最终,
$$A_{i+1,}=P_{i,}-A_{i,}=A_{i,}L-A_{i,}=A_{i,}(L-I)$$ 

这样就求出了 $A$的每一行。最后就可以用 $\by_{n+1}'=A\by_{n+1}=M(A\by_{n})$再加上初始值就可以只用加法计算多项式了。

最后还有一个问题,计算$L$。对于二项式系数有:
$$\binom{n}{k}=\binom{n-1}{k-1}+\binom{n-1}{k},\binom{n}{0}=\binom{n}{1}=\binom{n}{n}=1$$
这样可以按行计算出$L$。
\subsection{逼近理论}
给定一个系统,计算系统的极限,或者对它的值进行估计。

\subsubsection{级数展开法}
首先通过级数展开取主项,然后设法构造简单的迭代式如等差数列,再求和。有时展开后的项与要求的目标不一定相符,可能需要进行一些变换。级数展开的实质是求导,所以一般也可以用洛必达法则或者Stolz定理。

一些思路:
\begin{enumerate}
\item 迭代式里面没有$n$,但要求的极限有$n$,那么设想构造等差数列,求和就出现$n$了。
\end{enumerate}

\begin{example}
已知$x_{n+1}=\sin(x_n),x_n<3$,试计算
$$\lim_{n\rightarrow \infty} \sqrt{n}x_n$$
\end{example}
\begin{solution}
迭代的表达式里面没有$n$,但要求的极限里面有,因此必须首先将$n$与迭代过程联系起来。一种思路就是级数展开:
$$\sin x\approx x-\frac{1}{6}x^3+O(x^5)$$
那么有
\begin{empheq}{align*}
\frac{1}{x_{n+1}^2}&=\frac{1}{\sbra{x_n-\frac{1}{6}x_n^3+O(x_n^5)}^2}\\
&=\inv{x_n^2}\inv{\sbra{1-\frac{x_n^2}{6}+O(x_n^4)}^2}\\
&=\inv{x_n^2}+\inv{3}+O(x_n^2)
\end{empheq}
令$y_n=\inv{x_n^2}$,那么
\begin{empheq}{align*}
y_{n+1}&=y_n+\inv{3}+O\sbra{\inv{y_n}}\\
\frac{y_n+1}{n}&=\frac{n}{3}+O\sbra{\sum_{k=1}^n\inv{y_k}}
\end{empheq}
然后很容易得到原来的结果是$\sqrt{3}$。

本例还可以用Stolz定理的引理\ref{lem:Stoltz-l1}求解。

\begin{empheq}{align*}
\lim_{n\rightarrow \infty}\sbra{\inv{x_{n+1}^2}-\inv{x_n^2}}&=\lim_{n\rightarrow \infty}\sbra{\inv{\sin^2 x_n}-\inv{x_n^2}}\\
&=\lim_{x\rightarrow 0}\sbra{\inv{\sin^2x}-\inv{x^2}}\\
&=\inv{3}
\end{empheq}
立即可得原来的结果。
\end{solution}

\begin{example}
给定系统$x_1=\frac{1}{2},x_{n+1}=x_n-\frac{x_n^2}{16}$,证明
$$\frac{1}{n+1}\leq x_n\leq\frac{16}{n}$$
\end{example}
\begin{solution}
首先原命题相当于
$$n+1\geq \frac{1}{x_n}\geq \frac{n}{16}$$
这与上面的问题相似,对于原系统,取$\frac{1}{x_{n+1}}=\frac{1}{x_n-\frac{x_n^2}{16}}$,可以很容易得到$\inv{x_{n+1}}\approx \inv{x_n}+\frac{1}{16}$,即得到不等式的右边。

完整的证明可以按以下入手,首先证明$x_n$是单调递减的,再证明$1\geq \inv{x_{n+1}}- \inv{x_n}\geq \inv{16}$,等差数列相加就是原命题。
\end{solution}

\section{随机动力系统}

\subsection{稳定性}
\subsubsection{Stochastic Stablity Lemma}
\begin{lemma}[Stochastic Stablity Lemma]
给定常数$v_1,v_2,\mu>0, 0<\alpha<1$,如果存在一个随机过程$V(\xi_k)$满足以下条件:
$$v_1\|\xi_n\|^2\leq V(\xi_n)\leq v_2\|\xi_n\|^2$$
$$\E(V(\xi_n)|\xi_{n-1})-V(\xi_{n-1})\leq \mu-\alpha V(\xi_{n-1})$$
那么有
\begin{empheq}{equation}\label{Stochastic-Stablity-Lemma-result}
\E(\|\xi_n\|^2)\leq \frac{v_2}{v_1}\E(\|\xi_0\|^2)(1-\alpha)^n+\frac{\mu}{v}\sum_{i=0}^{n-1}(1-\alpha)^i
\end{empheq}
\end{lemma}
这个引理给出了二阶矩的上界,或者说通过$V$的界来控制变量的界,相当于从结果来反推原因的性质。
\section{混沌理论}
\subsection{基本概念}

