\subsection{泛函分析中的空间}
\subsubsection{各种空间的图景}
\begin{center}
\includegraphics[width=14cm]{figure/Functional Space - Overview.png}
\end{center}
\subsubsection{拓扑空间}
\begin{definition}[拓扑空间]\label{}
拓扑空间是一个集合$X$及其子集族$\tau$,满足以下条件:

\begin{enumerate}
	\item 空集和$X$属于$\tau$.
	\item $\tau$中任意多个集合的交和并均属于$\tau$.
\end{enumerate}

称子集族$\tau$为$X$的拓扑.$\tau$中的元素称为开集.开集的补集为闭集,集合可以是开的、闭的、非开非闭、既开又闭.

集合$\{\emptyset,X\}$构成平凡拓扑.

给定两个拓扑空间$X,Y$和一个映射$f\colon X\rightarrow Y$,如果$Y$中开集的原像是$X$中的开集,则称$f$连续.如果$X$中开集的像在$Y$中是开集,称$f$为开映射.

$X$到$Y$的所有连续映射记为$\mathcal{C}(X;Y)$.
映射$f\colon X\rightarrow Y$称为$X$到$Y$上的同胚,如果$f$是双射,且$f\in\mathcal{C}(X;Y),f^{-1}\in \mathcal{C}(X;Y)$.此时又称$X$与$Y$同胚.
\end{definition}

与子集族$\tau$相近的一个概念是滤子.

\begin{definition}[滤子]\label{}
设$X$为一集合,$\mathcal{F}$是$X$的非空子集族,如果:
\begin{enumerate}
\item $\mathcal{F}$中任意两个元素的交属于$\mathcal{F}$.
\item 如果$A\in\mathcal{F},A\subset B\subset X$,那么$B\subset \mathcal{F}$.
\end{enumerate}	

\end{definition}
第一个性质是指$\mathcal{F}$对交运算完备(但不像拓扑空间中一样涉及并运算),第二个性质类似于扩张.滤子的概念与$\sigma$代数十分相似.

拓扑空间的例子有:距离空间.

拓扑空间可以有连续性(指映射连续)、紧性、连通性.之前的定义中已经确定了连续性.

\begin{definition}[拓扑空间的紧性]\label{}
拓扑空间$X$是紧的,是指对$X$中任何一族开集$(O_i)_{i\in I}$,当$K\subset \bigcup_{i\in I}O_i$时,必存在$(O_i)_{i\in I}$的有限子族$(O_j)_{j\in J}$,使得$K\subset \bigcup_{j\in J}O_j$.
	
一种等价描述性:任何开覆盖均有有限子覆盖.

这种性质又称为Heine-Borel-Lebesgue性质.

在欧几里得空间中,这又等价于集合封闭且有界.
\end{definition}

开覆盖是指$\bigcup_{i\in I}O_i$,有限子覆盖是指$\bigcup_{j\in J}O_j$.

乍看上去,紧性似乎对每个拓扑空间都应该成立.因为既然$K\subset \bigcup_{i\in I}O_i$,即$K$属于交集,那么这个交肯定是由子集族中一些集合相交得到的.

问题核心就在这个“有限”上.比如取区间$(0,1]$,它不是紧的,因为它有开覆盖$\big\{(\frac{1}{n},1]\big\}_{n=1}^\infty$.但它的任意子覆盖必然不能覆盖区间$(0,1]$.但区间$[0,1]$就是紧的,前述的开覆盖并不能覆盖它.

紧空间的一些好处有:
\begin{enumerate}
\item 紧空间上的连续函数必然有最大最小值.
\end{enumerate}


\begin{definition}[拓扑空间的连通性]\label{}
拓扑空间$(X,\mathcal{O})$是连通的,是指$X$中既是开集又是闭集的子集只有$X$和$\emptyset$.

等价于$X$不能被分为两个不相交开集的并.

$X$的子集$A$称为连通的,是指$A$关于$X$在$A$上的诱导拓扑,它是一个连通的拓扑空间.

设$x,y\in X$,从$x$到$y$中的道路是指连续映射$[0,1]\rightarrow X$,它满足$\gamma(0)=x,\gamma(y)=y$.如果任意两个不同的点存在道路,称$X$弧连通.
\end{definition}

连通性的等价定义更容易理解——不能分为不相交开集的并.比如两个不相交的圆取并形成的集合,就不是连通的.连通性可以诱导中值定理.

\begin{definition}[Bolzano中值定理]\label{}
$X$为连通拓扑空间,$f\colon X\rightarrow \mathbb{R}$为连续函数,$a,b\in X$且$f(a)<f(b)$,则对任意给定的$y\in]f(a),f(b)[$,存在$x\in X$,使得$f(x)=y$.
\end{definition}

通俗地说,中值定理是指连通拓扑空间上的连续函数可以取一切中间值.

对于一个特定的集合可以构造很多拓扑,但很多是没有用的,如果引入更多的公理(或者说约束),就能定义更好性质的拓扑.常见的有分离公理、可数性公理.

\begin{definition}{分离定理}{}
$T_0$公理:任意不同的两点,存在某点的开邻域,不包含另外一点.

$T_1$公理:任意不同的两点,每个点存在开邻域,不包含另外一点.

$T_2$公理:任意不同的两点,每个点存在互不相交的开邻域.

满足$T_2$公理的空间又称$T_2$空间、豪斯多夫空间.$T_2$空间中每个序列不会收敛到两点以上的点.度量空间都是$T_2$空间.
\end{definition}

从$T_0$到$T_2$是递进的,越来越严格.比如$T_1$公理中,邻域不包含另外一点,但可以相交.

\subsubsection{距离空间}
\paragraph*{距离空间}就是集合加上距离函数。
\begin{definition}[距离空间]\label{}
设$X$是一个集合,函数$d:X\times X\rightarrow \mathbb{R}$满足下列性质:$\forall x,y,z\in X$,
\begin{empheq}{align*}
d(x,x)&=0,\text{且如果}x\neq y,\text{则}d(x,y)>0\\
d(x,y)&=d(y,x)\\
d(x,z)&\leq d(x,y)+d(y,z)
\end{empheq}
$(X,d)$为距离空间。
\end{definition}
\paragraph*{完备距离空间}每个Cauchy序列均在$X$中收敛的距离空间,在$X$中收敛需要满足两个条件,一是收敛,二是收敛的点也在$X$中。
\begin{definition}[Cauchy列]
记$(X,d)$为距离空间,$x_n\in X,n\geq 0$,如果$n\rightarrow \infty$,集合$\bigcup_{m=n}^\infty{x_m}$的直径收敛于0。

或者说$\forall \varepsilon>0,\exists n_0(\varepsilon)\geq 0$,使得当$m,n\geq n_0(\varepsilon)$时有$d(x_m,x_n)<\varepsilon$。
\end{definition}
\subsubsection{函数空间}
\begin{definition}[压缩映射]\label{concen-map}
设$(X,d)$为一距离空间,对映秀$f:X\rightarrow X$,如果存在常数$k$,使得$0<k<1$,且对任何$x,y\in X$,均有
$$d(f(x),f(y))\leq k d(x,y)$$
则称$f$为压缩映射.
\end{definition}

“压缩”指的是压缩距离.

\subsubsection{向量空间}
向量空间是集合配上加法与数乘。对于向量空间可以定义Hamel集,实质相当于最大独立集,和一般说的基相同。

\begin{definition}[赋范向量空间]\label{def::norm-vec-space}
设$X$为$\mathbb{K}$上的向量空间,$\mathbb{K}=\in\{\mathbb{R,C}\}$,设映射$\|\cdot\|:X\rightarrow\mathbb{R}$(范数)满足以下性质:
\begin{description}
\item[非负性] $\forall x\in X,\|x\|\geq 0$,且$\|x\|=0\implies x=0$。
\item[线性] $\forall \alpha \in \mathbb{K},x\in X, \|\alpha x\|=|\alpha|\|x\|$。
\item[三角不等式] $\forall x,y\in X,\|x+y\|\leq \|x\|+\|y\|$。
\end{description}
称$(X,\|\cdot\|)$为赋范向量空间。
\end{definition}
\paragraph*{赋范向量空间中的线性算子}对于算子也可以定义范数,但它与向量空间中的范数并不一样。
\begin{theorem}[算子的范数]\label{op-norm}
设$X,Y$是两个赋范向量空间,则
\begin{enumerate}[label=(\alph*)]
\item 由
$$\|\cdot\|:A\in\mathcal{L}(X;Y)\rightarrow \|A\|\coloneqq \sup_{x\neq 0}\frac{\|Ax\|}{\|x\|}$$
定义的映射是向量空间$\mathcal{L}$上的范数。且由定义有
$$\|Ax\|\leq \|A\|\|x\|$$
\item $A\in\mathcal{L}(X;Y)$可以等价定义为
\begin{empheq}{align*}
\|A\|&=\sum_{\|x\|\leq 1}\|Ax\|=\sum_{\|x\|<1}\|Ax\|=\sum_{\|x\|=1}\|Ax\|\\
&=\inv{r}\sum_{\|x\|\leq r}\|Ax\|=\inv{r}\sum_{\|x\|= r}\|Ax\|\\
&=\inf\{C>0\mid\forall x\in X,\|Ax\|\leq C\|x\|\}
\end{empheq}
\item 如果$X$是有限维空间,那么
$$\exists x_0\neq 0\in X,\|A\|\|x_0\|=\|Ax_0\| $$
\item 设$Z$为赋范向量空间,如果$A\in\mathcal{L}(X;Y),B\in\mathcal{L}(Y;Z)$,那么$BA\in\mathcal{L}(X;Z)$,且
$$\|BA\|\leq \|A\|\|B\|$$
那么,如果$A\in X$,则
$$\forall n\geq 0, \|A^n\|\leq \|A\|^n$$
\item 如果$A\in X$,则$A$的任意特征值$\lambda$满足$|\lambda|\leq \|A\|$。
\end{enumerate}
\end{theorem}
\subsubsection{Banach空间}
赋范向量空间$(X,\|\cdot\|)$是Banach空间,如果距离空间$(X,d)$完备,此处$d(x,y)=\|x-y\|$。相当于完备距离空间加上赋范向量空间。

\subsubsection{Hilbert空间}

