\chapter{空间几何}
%\input{投影}
\section{几何体}
\subsection{$n$维球}\label{ndsphere}
对于$n$维球,设半径为$R$,那么体积
$$V_n=\frac{\pi^{n/2}}{\Gamma(n/2+1)}R^n$$
$n-1$维球表面积为
$$S_{n-1}=\frac{\dif V_n}{\dif R}=\frac{2\pi^{n/2}}{\Gamma(n/2)}R^{n-1}$$
这里求表面积非常巧妙,将半径无穷小增加,则体积也增加,增加的比例于表面积有关。因此直接求导。

假如取单位球$R=1$,那么可以看出,当维数$n$增加时,体积趋向于0。这可能有点反直觉,因为高维球应该是“包含”低维球的,比如对于任意一个低维球中的点,只要把其它维度设成0,就可以变成高维球的点,为什么高维球的体积反而小了呢?

这主要是由于低维体在高维中,体积是0。比如二维平面在三维空间中,体积就是0。