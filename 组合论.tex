\chapter{组合论}
\section{基本原理}
\subsection{排列数与组合数}
\subsubsection{排列数}
\begin{definition}
从$n$种不物品中不放回取出$k$种物品,再将取出的物品进行排列,则最终可以得到不同的排列为排列数:
$$A(n,k)=\frac{n!}{(n-k)!}$$
\end{definition}

\subsubsection{组合数}
\begin{definition}
从$n\geq 0$种不物品中不放回取出$k$种物品,则不同的取法数量为组合数:
\begin{empheq}{align}
P(n,k)&=\binom{n}{k}=\frac{A(n,k)}{k!}=\frac{n!}{(n-k)!k!}\\
\binom{n}{0}=\binom{n}{n}=1,\quad n\geq 0
\end{empheq}
\end{definition}

\subsection{组合论等式}
\begin{empheq}{align*}
\binom{n}{k}+\binom{n}{k+1}=\binom{n+1}{k+1}
\end{empheq}

\subsection{鸽巢原理}
基本的表述是:将一个空间分成$n$类,那么从这个空间中取出$n+1$个元素,必然至少有两个元素处于同一类中.

使用鸽巢原理最重要的就是如何分类.对于不同的问题分类标准不同,比如对于数空间:正负、有理无理、奇偶等等.

\begin{example}
在二维平面上的整数格点中取5个点,则必然存在另一个整数格点,恰好位于这5个点中某两个点连线构成的线段上.

分类标准就是奇偶性,一个点的两个坐标奇偶性只有4种情况,就可以用鸽巢原理了.如(奇,奇),(奇,奇)的中点就是(偶,偶).
\end{example}

\section{组合论例子}
\subsection{样本问题}
\begin{example}
\textbf{(滑动窗口的采样重叠问题)}考虑滑动窗口,假设长度为$n$,滑动距离为1,那么每两个相邻的窗口,有$n-1$个重叠样本。假设从两个相邻的窗口中,每个窗口中取$m$个样本,问相同样本数量的分布。
\end{example}
\begin{solution}
显然地,全空间为$\binom{n}{m}^2$. $n$个样本可分为$1+(n-1)$个样本。取$m$个样本,分为两种情况:取或者不取前1个样本。则取两组样本时,有4种情况,一个一个列出来然后计算。

首先考虑最基本的情形:全不重叠。

假设第一组取了前1个样本,则第二组假如也取前1个,那么剩下$m-1$个要从$n-1-(m-1)=n-m$个中取。
\begin{empheq}{align*}
&\frac{\underbrace{\binom{n-1}{m-1}}_{\text{1取}}\left(\underbrace{\binom{n-m}{m-1}}_{\text{2取}}+\underbrace{\binom{n-m}{m}}_{\text{2不取}}\right)+\underbrace{\binom{n-1}{m}}_{\text{1不取}}\left(\underbrace{\binom{n-1-m}{m-1}}_{\text{2取}}+\underbrace{\binom{n-1-m}{m}}_{\text{2不取}}\right)}{\binom{n}{m}^2}\\
=&\frac{\underbrace{\binom{n-1}{m-1}}_{\text{1取}}\binom{n-m+1}{m}+\underbrace{\binom{n-1}{m}}_{\text{1不取}}\binom{n-m}{m}}{\binom{n}{m}^2}
\end{empheq}

再考虑有$k,k\leq m$个重叠的。

$k=m$比较显然。第一组只能在$n-1$个中取$m$个:
\begin{empheq}{align*}
\frac{\binom{n-1}{m}}{\binom{n}{m}^2}=\frac{n-m}{n}\bigg/\binom{n}{m}
\end{empheq}

对于其它情形。此时首先第一组任取,然后第二组取的时候先从第一组里面取$k$个,然后从剩下的取$m-k$个。第一组要分情况,假如第一组取了前1个,那么第二组只能从$m-1$个取$k$个,而且剩下的$m-k$个也分取或者不取两种情形,假如取了前1个,那么要从$n-1-(m-1)$个中$m-k-1$个;假如不取,从$n-1-(m-1)$个取$m-k$个。
\begin{empheq}{align*}
&\frac{\underbrace{\binom{n-1}{m-1}}_{\text{1取}}\binom{m-1}{k}\left(\underbrace{\binom{n-m}{m-k-1}}_{\text{2取}}+\underbrace{\binom{n-m}{m-k}}_{\text{2不取}}\right)+\underbrace{\binom{n-1}{m}}_{\text{1不取}}\binom{m}{k}\left(\underbrace{\binom{n-1-m}{m-k-1}}_{\text{2取}}+\underbrace{\binom{n-1-m}{m-k}}_{\text{2不取}}\right)}{\binom{n}{m}^2}\\
=&\frac{\underbrace{\binom{n-1}{m-1}}_{\text{1取}}\binom{m-1}{k}\binom{n-m+1}{m-k}+\underbrace{\binom{n-1}{m}}_{\text{1不取}}\binom{m}{k}\binom{n-m}{m-k}}{\binom{n}{m}^2}
\end{empheq}

以下给出一些算例:
\begin{enumerate}
\item 假设$n=1000,m=50$,则完全不重合的概率$P(0)\simeq 0.07$,前几个概率值如下:
\begin{empheq}{align*}
P(1)&\simeq 0.2\quad &P(2)&\simeq 0.2659\\
P(3)&\simeq 0.2259\quad & P(4)&\simeq 0.1378\\
P(5)&\simeq 0.0644 \quad & P(6) &\simeq 0.0239\\
P(7)&\simeq 0.0072 \quad & P(8) &\simeq 0.001856
\end{empheq}
$$\sum_{k=1}^{7}\simeq 0.86$$

可见,大概率只有少数几个样本会重合。
\end{enumerate}
\end{solution}

\begin{example}
\textbf{(随机采样的覆盖面)}$N$个样本,有放回取$n$个,问这些样本中不同样本个数$k,k\leq \min(N,n)$的分布。

训练模型时,经常是随机采样,通过这个问题,主要是看看需要抽多少次,才能使所有样本都被抽到的概率大于一定值。
\end{example}
\begin{solution}
全空间显然为$N^n$。
\end{solution}
\begin{example}\label{birthday_paradox}
\begin{theorem}[生日悖论]{}
	$N$为正整数,从中独立、随机选择$Q$个元素$x_1,\cdots,x_Q$,则
	$$\frac{Q(Q-1)}{4n}\leq P(\exists (i,j),x_i=x_j)=\frac{n\times(n-1)\times(n-Q+1)}{}\leq \frac{Q^2}{2N}$$
\end{theorem}

在上面的下界中,取$Q=\sqrt{2N}$,那么下界就接近$\fh$了。取1年365天,那么只要取$\sqrt{2\times 364}\approx 27$个人,其中有两个人同月同日生的概率就大于$\fh$。
\end{example}

