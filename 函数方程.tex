\chapter{函数方程}
函数方程就是以函数为未知变量的方程,积分方程、微分方程、变分法都可以视为函数方程。

\section{分析方法}
\subsection{改写函数方程的技巧}
\begin{empheq}{align}
&f(x)=\inv{1+x}\implies \inv{x}=\inv{f(x)}-1\\
\implies& f(3x)=\frac{3}{\inv{x}+3}=\frac{3f(x)}{1+2f(x)}\\
\implies & \inv{f(3x)}=\inv{3}\inv{f(x)}+\frac{2}{3}
\end{empheq}

\section{求解方法}
\subsection{离散法}
把函数用一个特定输入、输出构成的限表来逼近。
\subsubsection{一维函数方程}
给定一个函数方程:
$$F(y(x),x)=0,\quad x\in[a,b]$$
把定义域分割为$n$份,则有$n+1$个未知数待求解,有$n+1$个方程组:
\begin{empheq}{align*}
F(y_0, x_0)&=0\\
\cdots&\\
F(y_n, x_n)&=0
\end{empheq}
可以用数值方法求解。对于一个不在节点上的值$x$,可以用插值法求解$y(x)$。

\subsection{半数值方法}
\subsubsection{待定系数法}
结合函数逼近理论,取一系列有限的基函数$f_i(x)$,令$\tilde{y}=\sum_{i=1}^{n}a_if_i(x)$。然后优化参数$a_i$,这可以有多种选择,一是像离散法一样选择有限数量的点当成回归做。也可以用积分的方式:
\begin{empheq}{equation}
\min\quad \mathcal{L}=\inv{2}\int_{a}^{b}F(\tilde{y}(x),x)^2\dif x
\end{empheq}
距离函数也可以取绝对值。导数为:
\begin{empheq}{equation}
\pdv{\mathcal{L}}{a_k}=\int_{a}^{b}F(\tilde{y}(x),x)\left.\pdv{F}{a}\right|_x\dif x
\end{empheq}
实际应用时,发现优化非常困难,可能是问题本身高度非线性。不过可以用来检验优化算法的能力。

还有一个问题是,假如区间比较大,则可能需要非常高阶的级数。

\subsubsection{DJM方法}
一种半数值方法,结合了级数展开与不动点迭代。

对于方程
\begin{empheq}{equation}\label{functional-eq-djm-prob}
y(x)=f(x)+N(y(x))
\end{empheq}
假设一个解:
\begin{empheq}{equation}
y=\sum_{i=1}^{\infty}y_i(x)
\end{empheq}
则
\begin{empheq}{equation}
N(y)=N(\sum_{i=1}^{\infty}y_i(x))=N(y_0)+\sum_{i=1}^{\infty}\left[N\left(\sum_{j=0}^{i}y_j(x)\right)-N\left(\sum_{j=0}^{i-1}y_j(x)\right)\right]
\end{empheq}
代入原问题\cref{functional-eq-djm-prob}有:
$$\sum_{i=1}^{\infty}y_i(x)=f(x)+N(y_0)+\sum_{i=1}^{\infty}\left[N\left(\sum_{j=0}^{i}y_j(x)\right)-N\left(\sum_{j=0}^{i-1}y_j(x)\right)\right]
$$
对比系数,可以得到一个迭代算法:
\begin{empheq}{align}
y_0&=f(x)\\
y_1&=N(y_0)\\
y_2&=N(y_0+y_1)-N(y_0)\\
\cdots&\\
y_{n+1}&=N(y_0+\cdots+y_n)-N(y_0+\cdots+y_{n-1})
\end{empheq}
实际计算时可以选择一个特定的迭代次数。$y_0$也可以根据具体情况,取$f$的一部分,比如假设$f=f_0+f_1$,则有:
$$\sum_{i=1}^{\infty}y_i(x)=f_0+f_1+N(y_0)+\sum_{i=1}^{\infty}\left[N\left(\sum_{j=0}^{i}y_j(x)\right)-N\left(\sum_{j=0}^{i-1}y_j(x)\right)\right]
$$
也可以有迭代算法:
\begin{empheq}{align}\label{functional-eq-djm-rec2}
y_0&=f_0\\
y_1&=f_1+N(y_0)\\
y_2&=N(y_0+y_1)-N(y_0)\\
\cdots&\\
y_{n+1}&=N(y_0+\cdots+y_n)-N(y_0+\cdots+y_{n-1})
\end{empheq}
当然$y_1=f_1,y_2=N(y_0)$也是可以的。

\subsection{求解实例}
\subsubsection{积分方程}
\paragraph*{问题}
\begin{example}
给定一非线性沃尔泰拉方程:
$$y(x)=e^x-\frac{xe^{3x}}{3}+\frac{x}{3}+\int_0^x xy^3(t)\dif t,\quad x\in [-1,3]$$
\end{example}
积分方程用离散法不太好解,假设取$x_0,\cdots, x_n$,假设积分用梯形法则逼近,则需要0在参数中才好用,具体计算的时候,还要判断应该用哪些块加入计算,不太方便。积分方程适合用半数值方法。

\paragraph*{级数逼近法}用一个5次多项式逼近,外层有积分,内层也有积分:
\begin{empheq}{equation}
\min\quad \mathcal{L}=\int_{-1}^{3} \left(\tilde{y}(x)-\left(e^x-\frac{xe^{3x}}{3}+\frac{x}{3}+\int_0^x x\tilde{y}^3(t)\dif t\right)\right)^2\dif x
\end{empheq}
强调上式不等于:
\begin{empheq}{equation}
\min\quad \mathcal{L}'=\int_{-1}^{3}\dif x\int_0^x \left(\tilde{y}(x)-\left(e^x-\frac{xe^{3x}}{3}+\frac{x}{3}+ x\tilde{y}^3(t)\dif t\right)\right)^2\dif t
\end{empheq}
注意积分的范围,这个积分可以是负值,虽然被积函数是正的。但原来的积分必然是正值。

现在计算导数:
\begin{empheq}{align*}
\pdv{\mathcal{L}}{a_k}&=\int_{-1}^{3} \left(\tilde{y}(x)-\left(e^x-\frac{xe^{3x}}{3}+\frac{x}{3}+\int_0^x x\tilde{y}^3(t)\dif t\right)\right)\left(x^k-\int_0^x3x\tilde{y}^2(t)t^k\dif t\right)\dif x\\
\pdv{\mathcal{L}}{a_k,a_j}&=\int_{-1}^{3} \left[\left(x^j-\int_0^x 3x\tilde{y}^2(t)t^j\dif t\right)\left(x^k-\int_0^x3x\tilde{y}^2(t)t^k\dif t\right)\right.\\
&\quad \left.-\left(\tilde{y}(x)-\left(e^x-\frac{xe^{3x}}{3}+\frac{x}{3}+\int_0^x x\tilde{y}^3(t)\dif t\right)\right)\int_0^x 6x\tilde{y}(t)t^{k+j}\dif t\right]\dif x
\end{empheq}


\paragraph*{DJM方法}按\cref{functional-eq-djm-rec2}的迭代过程如下:
\begin{empheq}{align*}
y_0&=e^x\\
y_1&=-\frac{xe^{3x}}{3}+\frac{x}{3}+N(y_0)=-\frac{xe^{3x}}{3}+\frac{x}{3}+\int_0^x xe^{3t}\dif t=0\\
y_2&=N(y_0+y_1)-N(y_0)=0
\cdots&\\
y_n&=0,\quad n\geq 1
\end{empheq}
因此$y=e^x$,恰为精确解。


\section{积分方程}