\section{极值理论}
主要考查诸如局部极值、最大与最小值的分布一类的问题。
\subsection{局部极值的分布}

\subsubsection{图上随机函数的局部极值数量分布}
\cite{distribution-of-local-minima-of-graph}指出:
\begin{theorem}{图上随机函数的局部极值数量分布}
给定一个$d$图$G(V,E)$,即每个节点有$d$个相邻节点,又令$F$是一个$G$上的随机函数,且在每个节点处的值是独立分布的(只限定连续,不限定分布的具体类型)。取$W$为$F$的局部极值数量,又取$\E(W)=\mu,\Var (W)=\sigma^2$,则
$$\E (W)=\frac{|V|}{d+1}$$
且对于任意正数$w$,有
$$\left|\Prob(W\leq w)l-\Phi\left(\frac{w-\mu}{\sigma}\right)\right|\leq\frac{C}{\sqrt{\sigma}}$$
\end{theorem}

这个定理是在说,局部极值的分布接近正态分布。但文章作者也认为这个上界不是紧的,最优的界(收敛速度)可能与$\inv{\sigma}$成正比,而不是$\frac{1}{\sqrt{\sigma}}$。

以下给出一些应用的例子:
\begin{enumerate}
	\item 对于一个$n\times n$的二维网格,假设边界是周期性的,则每个点有4个相邻节点,$d=4$,那么$|V_n|=n^2$,
	$$\E(W_n)=\frac{n^2}{5}$$
	
\end{enumerate}