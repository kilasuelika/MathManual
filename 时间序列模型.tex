\chapter{时间序列模型}
\section{时间序列的特征}
\subsection{自相关}

\subsection{谱密度}

\section{序列检验}
\subsection{自相关}
\subsubsection{Ljung Box Test}
$$H_0:\text{ 模型没有自相关 }$$

如果$p$值显著,拒绝零假设,表示模型存在自相关。

\subsubsection{}

\subsection{正态性}

\section{自回归模型}
\subsection{ARIMAX}
\subsubsection{模型结构}
基本模型如下:
\begin{empheq}{equation}
\Delta^d y_t=X_t\bm{\beta}\sum_{k=1}^p\phi_k \Delta^d y_{t-k}+\sum_{k=1}^q\gamma_k \varepsilon_{t-k}+\varepsilon_t
\end{empheq}
$\Delta^d$表示高阶差分,比如
\begin{empheq}{align}
\Delta^0 y_t&=y_t\\
\Delta^1 y_t&=y_t-y_{t-1}\\
\Delta^2 y_t&=(y_t-y_{t-1})-(y_{t-1}-y_{t-2})=y_t-2y_{t-1}+y_{t-2}=\Delta^1 y_t-\Delta^1 y_{t-1}
\end{empheq}

利用算子代数,可以将ARIMAX表示为
\begin{empheq}{equation}
\left(1-\sum_{i=1}^{p}\phi_iL_i\right)(1-L)^dy_t=X_t\bm{\beta}+\left(1-\sum_{i=1}^{p}\gamma_iL_i\right)\varepsilon_t
\end{empheq}
$(1-L)^d$中的$d$就是$d$次方,比如
$$(1-L)^2=1-2L_1+L_2$$
对应
$$y_t-2y_{t-1}+y_{t-2}$$

\subsubsection{一阶自回归}
\paragraph*{模型结构}
模型为
$$y_t=ay_{t-1}+\varepsilon_t$$
现在把它展开:
\begin{empheq}{align*}
	y_t&=a(ay_{t-2}+\varepsilon_{t-1})+\varepsilon_t\\
	&=a^ty_0+\sum_{k=0}^{t-1} a^{k}\varepsilon_{t-k}
\end{empheq}
由于残差的期望是0,所以上式第二项期望为0,则长期来看$\E(y_t)=a^ty_0\rightarrow 0$。现再加上常数项$c$,则
模型为
$$y_t=ay_{t-1}+c+\varepsilon_t$$
\begin{empheq}{align*}
y_t&=a^ty_0+\sum_{k=0}^{t-1} a^{k}c+\sum_{k=0}^{t-1} a^{k}\varepsilon_{t-k}\\
&=a^ty_0+c\frac{1-a^t}{1-a}+\sum_{k=0}^{t-1} a^{k}\varepsilon_{t-k}
\end{empheq}
注意由于$c,\varepsilon$的地位相同,所以不必重新推导,只需要替换一下即可。在上式中$y_t\rightarrow \frac{c}{1-a}$,即便在历史的每一项中都加上$c$,但由于远距离样本的影响会衰减,所以整体上序列还是会收敛的。

\paragraph*{冲击响应}
记$y_t^*=ay_{t-1}^*=a^ty_0$,这是无冲击下的模型。再定义
\begin{empheq}{align*}
\tilde{y}_t&=y_t-y_t^*\\
&=ay_{t-1}+\varepsilon_t-ay_{t-1}^*\\
&=a(y_{t-1}-y_{t-1}^*)+\varepsilon_t\\
&=a\tilde{y}_{t-1}+\varepsilon_t
\end{empheq}
边界条件为$\tilde{y}_{0}=0$。

假设在$k$时刻有一个冲击$\varepsilon_k$,其它时间没有冲击,则$\tilde{y}_k=\varepsilon_t$。那么$\tilde{y}_{k+1}=a\varepsilon_t$,当$|a|<1$时,最终$\tilde{y}_{n}\rightarrow 0$,即最终会回到原来的路径。

一个有意思的现象是$1>a>0,\varepsilon_t>0$时,$\tilde{y}_{n}\geq 0$,且为单调递减序列。但假如$-1<a<0$,则$\tilde{y}_{k}>0,\tilde{y}_{k+1}<0,\tilde{y}_{k+2}>0,\cdots $这样就形成波动。


\section{随机波动率模型}

\subsection{ARCH族}
\subsubsection{基本结构}
ARCH模型主要用于描述残差的自相关性,均值可以使用其它的任意模型。基本结构如下:
\begin{empheq}{align}
y_t&=f_t+\varepsilon_{t}\mtag{均值方程}\\
\varepsilon_t&=\sigma_tz_t\\
\sigma_t^2&=g(\varepsilon_{t-1}^2,\cdots,\sigma_{t-1}^2,\cdots)\\
z_t&\sim N(0,1)
\end{empheq}
$g$可以使用不同的模型,这里的正态分布假定也可以使用其它的分布,比如$t$分布。注意到在上面的模型中
$$\varepsilon_t\sim N(0,\sigma_t^2)\equiv N(0,g_t^2)$$

$f_t$同样可以由不同的模型来给出,有时直接取常数。

可计算条件期望为
\begin{empheq}{align*}
\E_t(\varepsilon_{t})&=\E_t(\sigma_t^2)
\end{empheq}

实际建模时,可以独立建模均值方程,然后计算残差序列,对残差序列建立模型。也可以联合起来估计。
\subsubsection{GARCH}
\paragraph*{模型结构}取$f$为线性,GARCH($p,q$)模型如下:
$$\sigma_t^2=\omega+\sum_{k=1}^{p}\beta_k\sigma_{t-k}^2+\sum_{k=1}^{q}\alpha_k\varepsilon_{t-k}^2$$

\paragraph*{条件分布}
\begin{empheq}{align*}
\E_{t+1}&=\omega+\sum_{k=1}^{p}\beta_k\E_t(\sigma_{t-k}^2)+\sum_{k=1}^{q}\alpha_k\E_t(\varepsilon_{t-k}^2)\\
&=\omega+\sum_{k=1}^{p}\beta_k\sigma_{t-k}^2+\sum_{k=1}^{q}\alpha_k\sigma_{t-k}^2
\end{empheq}
注意由于是取条件分布,所以在$t+1$时刻,$\sigma_+t$是已知的。类似地,计算$\E_t(\sigma_{t+2}^2)$时需要用到$\E_t(\sigma_{t+1}^2)$。以GARCH(1,1)为例:
\begin{empheq}{align*}
\E_t(\sigma_{t+1}^2)&=\omega+(\beta+\alpha)\sigma_t^2\\
\E_t(\sigma_{t+2}^2)&=\E_t(\omega+\beta\sigma_{t+1}^2+\alpha\varepsilon_{t+1}^2)\\
&=(1+(\beta+\alpha))\omega+(\beta+\alpha)^2\sigma_t^2\\
\E_t(\sigma_{t+n}^2)&=\omega\frac{1-(\beta+\alpha)^n}{1-(\beta+\alpha)}+(\beta+\alpha)^n\sigma_t^2
\end{empheq}
易知稳定性条件为
$$\beta+\alpha<1$$
极限为
$$\lim_{n\rightarrow \infty}\E_t(\sigma_{t+n}^2)=\frac{\omega}{1-(\beta+\alpha)}$$

\paragraph*{MLE估计}根据残差的分布,可以计算似然函数为
$$L(\btheta;\by)=L(\btheta;y_0)\prod N(y_t-f_t,\sigma_t^2)$$
前一项为初始值,通常取无条件分布,有时也直接忽略掉这一项。实际计算时首先计算所有$\varepsilon_t$,然后再迭代计算$\sigma_t^2$。

仍以GARCH(1,1)模型为例:
\begin{enumerate}
	\item 计算残差序列$\{\varepsilon_t\}_0^N$
	\item 猜测一个$\sigma_0^2$,通常取残差的方差。然后根据递推关系计算$\{\sigma_t\}_1^N$。
	\item 优化似然函数:
	\begin{empheq}{align*}
		\ln\mathcal{L}(\omega,\beta,\alpha)&=\sum_{t=1}^{N}\ln(N_{\text{PDF}}(\varepsilon_t;0,\sigma_t^2))\\
		&=\sum_{t=1}^{N}\left(-\inv{2}\ln 2\pi-\frac{\varepsilon_t^2}{2\sigma_t^2}-\inv{2}\ln\sigma_t^2\right)
	\end{empheq}
\end{enumerate}

如果使用自动微分工具,且联合均值方程,假设均值方程为常数,则现在参数为$(c,\omega,\beta,\alpha)$,要采用循环的方式, 但$\sigma_0^2$不易取得,可以先单独估计均值方程,计算一个$\sigma_0^2$,再循环:
\begin{enumerate}
	\item $L=0$
	\item $t=1\cdots N$:
	\begin{enumerate}
		\item 计算$\varepsilon_t$,计算样本的似然函数$L_t$ 
		\item $L+=L_t$
	\end{enumerate}
\end{enumerate}
\subsubsection{NAGARCH}
即Nonlinear Asymmetric GARCH(1,1)模型:
$$\sigma_t^2=\omega+\alpha(\varepsilon_{t-1}-\theta\sigma_{t-1})^2+\beta\sigma_{t-1}^2$$
$\theta$一般为正值,模型用来刻画杠杆效应,即负收益率(对应$\varepsilon_{t-1}$为负值)对波动的影响大于正收益率。

为保证波动率为正,要求$\omega>0,\alpha,\beta\geq 0$。
\subsubsection{IGARCH}
Integrated GARCH,GARCH的系数限制版本:
$$\sum_{k=1}^{p}\beta_k+\sum_{k=1}^{q}\alpha_k=1$$

\subsubsection{EGARCH}
指数GARCH:
\begin{empheq}{align*}
\ln\sigma_t^2&=\alpha_0+\sum_{k=1}^{p}\beta_k\ln\sigma_{t-k}^2+\sum_{k=1}^{q}\alpha_kg(z_{t-k})\\
g(z_{t-k})&=\theta z_{t}+\gamma(|z_t|-\E(|z_t|))
\end{empheq}
$z_t$通常可以取标准正态分布,此时$\E(|z_t|)=\sqrt{\frac{2}{\pi}}$。

\subsubsection{GARCH-in-Mean}
在均值方程中添加一个$\sigma_t$项:
$$y_t=f+\lambda\sigma_t+\varepsilon_t$$

\subsubsection{TGARCH}
对标准差建模:
$$\sigma_t=K+\delta\sigma_{t-1}+\alpha_1\varepsilon_{t-1}^++\alpha_1^-\varepsilon_{t-1}^-$$
其中$\varepsilon_{t-1}^+=\max(\varepsilon_t,0)$,$\varepsilon_{t-1}^-$类似,同样用来刻画正负收益率的非对称性。