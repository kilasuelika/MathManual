\chapter{线性规划}
\section{基本概念}
\subsection{标准模型}
线性规划的标准模型是
\begin{empheq}{align}\label{lp-noninteger-standard}
\min\ & \bm{c}^T\bx\\
\text{s.t. }&A\bx=\bm{b}\\
&\bx\geq \bm{0}
\end{empheq}

对于非标准模型,可以通过松弛变量、取反等方法转换为标准模型。
\begin{enumerate}
\item 极大化$\max\ \bm{c}^T\bx$转换为$\min\ -\bm{c}^T\bx$。
\item 小于等于约束条件$\bm{a}^T\bx\leq b$,转换为$$\bm{a}^T\bx+x_k=b,x_k\geq 0$$
意为一个数如果小于另一个数,则前者要加上一个正数才会等于后一个数。

类似地,大于等于约束条件$\bm{a}^T\bx\geq b$转换为$-\bm{a}^T\bx\leq -b$,然后改写为
$$-\bm{a}^T\bx+x_k=-b,x_k\geq 0$$
或者
$$\bm{a}^T\bx-x_k=b,x_k\geq 0$$
\item $\bx$无界条件可以引入新的变量
$$\by=\begin{bmatrix}
\by_1\\\by_2
\end{bmatrix}\geq\bm{0}$$
令$\bx=\by_1-\by_2$,于是约束条件$A\bx=\bm{b}$可以改写为
$$\begin{bmatrix}
A & -A
\end{bmatrix}\begin{bmatrix}
\by_1 \\\by_2
\end{bmatrix}\implies\bm{b}=B\by=\bm{b}$$
目标函数现在是
$$\begin{bmatrix}
\bm{c}^T & -\bm{c}^T
\end{bmatrix}\by$$
此时变量数会加倍。
\item 为了容易得到初始基可行解,一般需要保证$\bm{b}\geq 0$,需要使用上面的技巧进行操作。
\end{enumerate}
%\section{存在性定理}

\subsection{对偶原理}
不等式线性约束问题
\begin{empheq}{equation}\label{lp-dual-1-standard}
		\begin{aligned}
	\min\ & \bm{c}^T\bx\\
	\text{s.t. }&A\bx\geq\bm{b}\\
	&\bx\geq \bm{0}
\end{aligned}
\end{empheq}
注意这个问题不是标准模型。其对偶问题是:
\begin{empheq}{equation}\label{lp-dual-2-standard}
		\begin{aligned}
	\max\ & \bm{b}^T\by\\
	\text{s.t. }&A^T\by\leq\bm{c}\\
	&\by\geq \bm{0}
\end{aligned}
\end{empheq}

带等式、无界约束线性规划问题:
\begin{empheq}{equation}\label{glp-dual-1-standard}
		\begin{aligned}
	\min\ & \bm{c}_1^T\bx_1+\bm{c}_2^T\bx_2\\
	\text{s.t. }&A_{11}\bx_1+A_{12}\bx_2=\bm{b}_2\\
	& A_{21}\bx_1+A_{22}\bx_2\geq\bm{b}_1\\
	&\bx_1\geq \bm{0}
		\end{aligned}
\end{empheq}
注意$\bx_2$没有有界约束。它的对偶问题是:
\begin{empheq}{equation}\label{glp-dual-2-standard}
	\begin{aligned}
	\max\ & \bm{b}_1^T\bm{w}_1+\bm{b}_2^T\bm{w}_2\\
	\text{s.t. }&A_{11}^T\bm{w}_1+A_{12}^T\bm{w}_2\geq \bm{c}_1\\
	& A_{21}\bm{w}_1+A_{22}^T\bm{w}_2=\bm{c}_1\\
	&\bm{w}_2\geq \bm{0}
		\end{aligned}
\end{empheq}
可以看出$=$与$\geq$互换,目标函数系数$\bm{c}$与$\bm{b}$互换,$\min$与$\max$互换。根据对偶原理:
\begin{enumerate}
\item \ref{glp-dual-1-standard}无界$\implies$\ref{glp-dual-2-standard}无可行解。
\item \ref{glp-dual-1-standard}无可行解$\implies$\ref{glp-dual-2-standard}无界或者无可行解。
\end{enumerate}
\section{单纯形法}
\subsection{基本算法}
\subsubsection{计算步骤}
确实一个初始基本可行解,记初始基为$B$。执行以下过程:
\begin{enumerate}
\item 解$B\bx_B=\bm{b}$,求得$\bx_B=B^{-1}\bm{b}=\bar{\bm{b}}$,取$\bx_N=\bm{0}$,计算目标函数值$\bm{c}_B^T\bx_B$。
\item 求单纯形乘子$\bm{w}$,解$\bm{w}B=\bm{c}_B$,有$\bm{w}=\bm{c}_BB^{-1}$,对于所有非基变量,计算判别数$z_j-c_j=\bm{w}^T\bm{p}_j=c_j$,令
$$z_k-c_k=\max \{z_j-c_j\}$$
如果全部判别数小于等于0,则停止。否则,进入下一步
\item 解$B\bm{y}_k=\bm{p}_k$,有$\by_k=B^{-1}\bm{p}_k$,假如$\by_k\leq \bm{0}$,即分量全为非正数,则停止,{\heiti 不存在}有限最优解,或者无界。否则进行下一步:
\item 确定下标$r$,使
$$\frac{\bar{b_r}}{y_{rk}}=\min\left\{\frac{\bar{b_i}}{y_{ik}}\rvert y_{ik}>0\right\}$$
$x_{B_r}$为离基变量,$x_k$为进基变量,用$\bm{p}_k$替换$\bm{p}_{B_r}$,得到新的基矩阵,再重复以上迭代。
\end{enumerate}

单纯形法的实质是每次求一个改进的解,使得函数值下降,且满足约束条件。由于约束条件的个数通常少于变量个数(否则可行解可能只有一个,直接求方程组即可),因此可以分为基变量与非基变量。判别数就指明了下降的量。比如
\begin{empheq}{align*}
f&=\bm{c}^T\bx=\begin{bmatrix}
	\bm{c}_B^T&\bm{c}_N^T
\end{bmatrix}\begin{bmatrix}
\bx_B\\\bx_N
\end{bmatrix}\\
&=\bm{c}_B^T\bm{x}_B+\bm{c}_N\bm{x}_N\\
&=\bm{c}_B(B^{-1}\bm{b}-B^{-1}N\bx_N)+\bm{c}_N\bx_N\\
&=\bm{c}_BB^{-1}\bm{b}-(\bm{c}_BB^{-1}N-\bm{c}_N)\bx_N\\
&=f_0-\sum (\bm{c}_BB^{-1}\bm{p}_j-c_j)x_j\\
&=f_0-\sum (z_j-c_j)x_j
\end{empheq}
$z_j-c_j$就是判别数,由于$x_j$大于0,那么取正判别数对应的变量进行调整,就可以使函数值下降。

一般情况下,基变量是新增加的变量(人工变量),约束矩阵中人工变量的系数为1,初始的基解是就是$\bm{b}$,非基变量为目标函数中的变量,此时判别数为负目标函数系数。但在有的情况下,是不一样的。但也可以取人工变量的系数为-1,主要希望保证$\bm{b}\geq \bm{0}$。


此外,即使保证了$\bm{b}\geq \bm{0}$,基变量也不一定就是人工变量。可以参考问题\ref{base-non-new}。

所以初始的基解有可能需要手动计算,不能直接得到。建立初表这样建立:
\begin{enumerate}
\item 基变量的约束矩阵是单位阵。
\item $\bm{b}\geq 0$。
\end{enumerate}

假设$\bm{b}$中存在0分量,则可能出现矩阵$B$不可逆的情况,称为退化。如果不存在,则一定不会出现不可逆。

\subsubsection{性质}
\begin{enumerate}
\item 假如某个非基变量的检验数为0,则有无穷多个解。
\item 假如进基变量的约束矩阵列全为负,则无下界。
\end{enumerate}
\subsection{单纯形表}
基本算法不便于手动操作,可以将它用表格表示出来,便于手动操作。

\subsubsection{基变量为人工变量}
\begin{example}
求解线性规划标准模型\eqref{lp-noninteger-standard}的单纯形表法如下。假设问题是:
\begin{empheq}{align}
	\min\ & \begin{bmatrix}
		1 & -2 &1 
	\end{bmatrix}\bx\\
	\text{s.t. }&\begin{bmatrix}
		1& 1&-2&1 & & \\
		2&-1&4& & 1& \\
		-1&2&-4& & & 1
	\end{bmatrix}\bx=\begin{bmatrix}
		10\\8\\4
	\end{bmatrix}\\
	&x_j\geq 0
\end{empheq}
\end{example}

\begin{solution}

构造初始单纯形表为
\begin{longtable}{c|cccccc|c}
	      & $x_1$ & $x_2$ & $x_3$ & $x_4$ & $x_5$ & $x_6$ &   \\ \hline
	& 1 & -2 & 1 & 0& 0 & 0&      \\ \hline
	$x_4$ &   1   &  1     &   -2    &  1    &   0    &  0     & 10  \\
	$x_5$ &   2   &   -1    &  4    &    0   &   1    &   0    &  8 \\
	$x_6$ &  -1   &     2  &   -4   &    0   &   0    &   1   & 4 \\ \hline
	   判别数   &  -1   &   2   &  -1   &   0   &   0   &   0   & 0
\end{longtable}
第一列的$x_4,x_5,x_6$称为基变量。第二大行就是目标函数中的系数。第三大行就是$A,\bm{b}$。第四大行为判别数$z_j-c_j=\bm{c}_B^TB^{-1}\bm{p}_j-c_j$,$B$是基变量$x_4,x_5,x_6$,$\bm{c}_B$是目标函数中基变量对应的系数,初始时为$\bm{0}$,$B$就是约束条件中基变量对应的系数矩阵,初始时就是单位阵$I$,$\bm{p}_j$是非基变量对应的约束矩阵列。容易看出,初始时,非基变量的判别数就是$-c_j$,基变量的判别数是0。

此外,初始解就是$[0,0,0,10,8,4]$,也就是非基变量是0,基变量就是$\bm{b}$,所以表中最后一个元素表示函数值0。

单纯形法的思路就是每次选一个基变量进行调整,直到达到最优。建表如下:
\begin{longtable}{c|c|cccccc|c}
&	& $x_1$ & $x_2$ & $x_3$ & $x_4$ & $x_5$ & $x_6$ &   \\ \hline
&& 1 & -2 & 1 & 0& 0 & 0&      \\ \hline
10/1=10&	$x_4$ &   1   &  1     &   -2    &  1    &   0    &  0     & 10  \\
&	$x_5$ &   2   &   -1    &  4    &    0   &   1    &   0    &  8 \\
4/2=2&	$x_6$ &  -1   &     \fbox{2}  &   -4   &    0   &   0    &   1   & 4 \\ \hline
&	判别数   &  -1   &   \circled{2}   &  -1   &   0   &   0   &   0   & 0
\end{longtable}
首先判别数中最大的是2,于是取第2列作为主列,再计算$b_i/a_{2i},a_{2i}>0$中的最小值,于是取第3行2列中的元素作为主元,进行RREF,使得该列中这个元素变成1,其它元素变成0。建表如下:
\begin{longtable}{c|c|cccccc|c}
	 &       & $x_1$ &    $x_2$    & $x_3$ & $x_4$ & $x_5$ & $x_6$ &    \\ \hline
	 && 1 & -2 & 1 & 0& 0 & 0&      \\ \hline
	 & $x_4$ &   3/2   &      0      &  0   &   1   &   0   &   -1/2   & 8 \\
	 & $x_5$ &   3/2   &     0      &   2   &   0   &   1   &   1/2   & 10  \\
	 & \circled{$x_2$} &  -1/2   &  \circled{1}   &  -2   &   0   &   0   &   1/2   & 2  \\ \hline
	 &  判别数  &     & 0 &     &   0   &   0   &      & 
\end{longtable}
现在基变量是$x_4,x_5,x_2$,注意到$A_{32}$从2变成1,该列其它元素是0.现在计算判别数,基变量的判别数是0,所以只需要计算非基变量$x_1,x_3,x_6$的:
\begin{empheq}{align*}
\bz_N-\bm{c}_N&=\bm{c}_BB^{-1}N-\bm{c}_N\\
&=
\begin{blockarray}{ccc}
x_4& x_5& x_2\\
\begin{block}{[ccc]}
0 & 0 & -2\\
\end{block}
\end{blockarray}
\begin{blockarray}{ccc}
	x_4& x_5& x_2\\
	\begin{block}{[ccc]}
		1 & 0 & 0\\
		0 & 1& 0\\
		0& 0& 1\\
	\end{block}
\end{blockarray}
\begin{blockarray}{ccc}
	x_1& x_3& x_6\\
	\begin{block}{[ccc]}
		3/2 & 0 & -1/2\\
		3/2& 2 & 1/2\\
		-1/2& -2 &1/2\\
	\end{block}
\end{blockarray}-\begin{blockarray}{ccc}
x_1& x_3& x_6\\
\begin{block}{[ccc]}
	1 & 1 & 0\\
\end{block}
\end{blockarray}\\
&=\begin{blockarray}{ccc}
	x_1& x_3& x_6\\
	\begin{block}{[ccc]}
		0 & 3 & -1\\
	\end{block}
\end{blockarray}
\end{empheq}
此处$N$表示非基变量的约束系数矩阵,$\bm{c}_N$就是非基变量的目标函数系数。实际计算过程中,$B$矩阵一般是1,所以不需要算矩阵逆。于是第一轮迭代后的完整表是
\begin{longtable}{c|c|cccccc|c}
	&       & $x_1$ &    $x_2$    & $x_3$ & $x_4$ & $x_5$ & $x_6$ &    \\ \hline
	&& 1 & -2 & 1 & 0& 0 & 0&      \\ \hline
	& $x_4$ &   3/2   &      0      &  0   &   1   &   0   &   -1/2   & 8 \\
10/2=5	& $x_5$ &   3/2   &     0      &   \fbox{2}   &   0   &   1   &   1/2   & 10  \\
	& \circled{$x_2$} &  -1/2   &  1   &  -2   &   0   &   0   &   1/2   & 2  \\ \hline
	&  判别数  &  0   & 0 &  3   &   0   &   0   &   -1   &  -4
\end{longtable}
最后一个-4就是表示目标函数值,此时的解是$[x_4,x_5,x_2]=[8,10,2]$,目标函数值是$-2\times x_2=-4$。现在主元是$A_{23}$。再进行一步迭代后:
\begin{longtable}{c|c|cccccc|c}
	 &       & $x_1$ & $x_2$ & $x_3$ & $x_4$ & $x_5$ & $x_6$ &    \\ \hline
	 &       &   1   &  -2   &   1   &   0   &   0   &   0   &    \\ \hline
	 & $x_4$ &  3/2  &   0   &   0   &   1   &   0   & -1/2  & 8  \\
	 & $x_3$ &  3/4  &   0   &   1   &   0   &   1/2   &  1/4  & 5 \\
	 & $x_2$ &   1   &   1   &   0   &   0   &   1   &  1  & 12  \\ \hline
	 &  判别数  &   -9/4   &   0   &   0   &   0   &   -3/2   &  -7/4   & -19
\end{longtable}
判别数全为负数,停止,现在解是$[0, 12, 5, 8, 0, 0]$,目标函数值-19。

在单纯型表中,停止条件有两类:
\begin{enumerate}
\item 判别数全负,正常停止,有最优解。
\item 最大的判别数大于0,但系数矩阵中该列全小于0,原问题无界。因为是要选择系数矩阵中大于0的系数行,那么无法进行下一步。
\end{enumerate}

\end{solution}

\subsubsection{基变量为非人工变量}\label{base-non-new}
\begin{example}
\begin{empheq}{align}
	\min\ & \begin{bmatrix}
		4 & 6 &18 
	\end{bmatrix}\bx\\
	\text{s.t. }&x_1+3x_3\geq 3\\
	& x_2+2x_3\geq 5\\
	x_1,x_2,x_3\geq 0
\end{empheq}
\end{example}
\begin{solution}
如果按上一题的方法直接构造表:
\begin{longtable}{c|ccccc|c}
	& $x_1$ & $x_2$ & $x_3$ & $x_4$ & $x_5$  &   \\ \hline
	& 4 & 6 & 18 & 0& 0    &   \\ \hline
	$x_4$ &   -1   &  0     &   -3    &  1    &   0      & -3  \\
	$x_5$ &   0   &   -1    &  -2    &    0   &   1      &  -5 \\ \hline
	判别数   &  -4   &   -6   &  -18   &   0   &   0    & 0
\end{longtable}
这个表对应基解$(0,0,0,-3,-5)$,显然是错的,$x_4,x_5<0$,不满足条件。

正解是可以取$x_1,x_2$作为初始基变量,建表:
\begin{longtable}{c|ccccc|c}
	& $x_1$ & $x_2$ & $x_3$ & $x_4$ & $x_5$  &   \\ \hline
	& 4 & 6 & 18 & 0& 0    &   \\ \hline
	$x_4$ &   1   &  0     &   3    &  -1    &   0      & 3  \\
	$x_5$ &   0   &   1    &  2    &    0   &   -1      &  5 \\ \hline
	判别数   &  0   &   0   &  -6   &   -4   &   -6    & 42
\end{longtable}
判别数中,基变量的是0,而非基变量的计算如下:
\begin{empheq}{align*}
\begin{bmatrix}
	4& 6
\end{bmatrix}I^{-1}\begin{bmatrix}
3 & -1 & 0\\2 & 0 & -1
\end{bmatrix}-\begin{bmatrix}
18 & 0& 0
\end{bmatrix}=\begin{bmatrix}
-6 & -4 & -6
\end{bmatrix}
\end{empheq}
表对应的解是$(3,5,0,0,0)$,所以$f=42$。

\begin{example}
\begin{empheq}{align}
		\min\ & \begin{bmatrix}
			1 & 1 &11 
		\end{bmatrix}\bx\\
		\text{s.t. }&x_1+2x_2-2x_3\leq 0\\
		& -x_1+x_3\leq -1\\
		x_1,x_2,x_3\geq 0
\end{empheq}
\end{example}
\begin{solution}
\paragraph*{初次尝试}
建立初表:
\begin{longtable}{c|ccccc|c}
	& $x_1$ & $x_2$ & $x_3$ & $x_4$ & $x_5$  &   \\ \hline
	& 1 & 1 & 1 & 0& 0    &   \\ \hline
	$x_1$ &   1   &  2     &   -2    &  1   &   0      & 0  \\
	$x_2$ &   1   &   0    &  -1    &    0   &   -1      &  1 \\ \hline
	判别数   &  0   &   0   &  $-\frac{5}{2}$   &   $\inv{2}$   &   $-\inv{2}$    & 
\end{longtable}
取$x_1,x_2$为基变量,基解为$(0,1,0,0,0)$。如果选择$x_4$换$x_1$,则约束系数矩阵是
$$\begin{bmatrix}
2&0\\
0& 0
\end{bmatrix}$$
不可逆。问题在于基解是错的,不满足要求。

\paragraph*{第二次尝试}
取$x_1,x_3$为基变量,建立初表:
\begin{longtable}{c|ccccc|c}
	& $x_1$ & $x_2$ & $x_3$ & $x_4$ & $x_5$  &   \\ \hline
	& 1 & 1 & 1 & 0& 0    &   \\ \hline
	$x_1$ &   1   &  -2     &   0    &  -1   &   -2      & 2  \\
	$x_2$ &   0   &   -2    &  1    &    -1   &   -1      &  1 \\ \hline
	判别数   &  0   &   -5   &  0   &   -2  &   -3    & 
\end{longtable}
这里为了方便,已经把基变量的约束矩阵化成单位阵了。可以看出已经得到最优解$(2,0,1,0,0)$了。
\end{solution}
\end{solution}