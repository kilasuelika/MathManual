\chapter{证明问题的常见思路}
\section{通用思路}
第一,尝试对原问题进行转换.

第二,证明逆否命题.

第三,构造法:构造反例、正例.

第四,从部分到整体.

阅读证明时,也要从以上入手.

\section{证明存在性与不存在性}
存在性通常是指:存在某个对象,满足某种性质.证明存在性,只要能构造一个正例即可.证明不存在性,构造反例,或者假设存在,推出矛盾.

阅读已有的证明时,也要从以下入手:\circled{1}正例是什么,反例是什么;\circled{2}构造例子的过程;\circled{3}背后的直觉.

一些特殊的技巧:

\begin{enumerate}
\item 证明存在性,首先构造一类对象,部分地满足性质,再从这一类对象中选择一些特殊的对象,比如最大、最小,证明这些特殊对象恰好可以满足全部性质.在证明Hann-Banach定理\ref{hann-banach}时应用了这种思路,此处,“部分满足”是指函数的定义域非全集.从中选择极大元,它的定义域恰好是全集.

如果选择极大元,必然要求极大在元存在,有时需要Zorn引理.

这种思路基本上是从部分到整体的思路,类似地,我们可以分别构造不同类的对象,它们分别部分地满足不同的性质,再把这些对象合成一类,比如分段线性函数.

\end{enumerate}
\section{证明解集的完备性}

对于一个问题,求出了一组解,并不代表解是完备的,该问题可能存在其它解,需要证明求出的解就是原问题的解.

首先,最直观地从个数上,思路是证明\circled{1}原问题的解最多只有$n$个,\circled{2}已经求出了$n$个解.所以求出的解是完备的.

更抽象地看,“个数”是集合的一种性质,那么也可以从其它性质入手.\circled{1}原问题的解必然满足某种性质,\circled{2}满足某种性质的解只能是求出的解.所以解集是完备的.

\section{证明唯一性}
唯一性是很重要的性质,尤其是在数值计算中.

第一,可以假设存在两个不同的元素,证明这两个元素是相同的,比如差为0.

第二,在分析学中,对于迭代给出的算法,有时可以证明,假设存在一个真实的值,从任意一个初始值出发,会收敛到相同的真实值;或者对于任意不同的初始值,分别迭代后,会趋于同一个值.

\section{扩张与数学归纳法}
对于少数事物间的联系与性质,将它推广到更多的事物。通常与数学归纳法密切相关。

\chapter{证明不等式}
\section{导数法}
构造一个函数,再求导得到单调性,是一种基本思路.

有时对于一些代数函数,我们设想可以用泰勒展开,比较最高次项的符号,这样可以得到在展开点附近的导数符号.但是它不能说明整个区间上的符号.所以另一方式就是求高阶导数.这两种方式基本上是一样的,比如两个函数,如果泰勒展开相减后消去了一阶项,只有二阶.那么求二阶导再相减后,一阶项也被消去了.
