\subsection{Korovkin定理}
\subsubsection{基本定理}
\begin{theorem}[Korovkin定理]{}
设$(K,d)$为紧距离空间,函数$\varphi\in\mathcal{C}[0,\infty]$满足$\forall t>0,\phi(t)>0$。对$\forall x\in K$,定义$\psi_x\in\mathcal{C}(K)$:
$$\psi_x(y)=\varphi(d(x,y)),y\in K$$

(不妨称为伪距离函数)。设线性算子$A_n:\mathcal{C}(K)\rightarrow \mathcal{C}(K)$的序列满足以下三个性质:
\begin{enumerate}
\item 保非负性。$A_n$保持非负,即假如$f\in\mathcal{C}(K)$,并且$\forall x\in K, f(x)\geq 0$,则
$$\forall x\in K, A_nf(x)\geq 0$$
\item 逼近常数1函数。对定义为$\forall x\in K,f_0(x)=1$的函数$f_0\in\mathcal{C}(K)$,有
$$\lim_{n\infty} \|f_0-A_nf_0\|=0$$
\item 局部逼近零函数。
$$\lim_{n\rightarrow \infty}\|f-A_nf\|=0$$
\end{enumerate}

那么$\forall f\in\mathcal{C}(K)$,有
$$\lim_{n\rightarrow \infty}\left(\sup_{x\in K}(A_n\phi_x)(x)\right)=0$$
\end{theorem}

本定理是证明了如果算子能够逼近某些特殊函数,则它能逼近空间中的任意函数。

按照极限的标准证法,需要证明的目标是:
$$\forall f\in\mathcal{C}(K),\varepsilon>0,\exists n_0=n_0(f,\varepsilon)\geq 0,\forall n\geq n_0,\sup_{x\in K}|(A_nf)(x)-f(x)|\leq \varepsilon$$

现在运用三角不等式:
\begin{empheq}{align*}
\forall x\in K,n\geq 0,|A_nf(x)-f(x)|&=|A_nf(x)-f(x)A_nf_0(x)+f(x)A_nf(0)(x)-f(x)f_0(x)|\\
&\leq |A_nf(x)-f(x)A_nf_0(x)|+|f(x)(A_nf_0-f_0)(x)|
\end{empheq}
注意到上界的第二项对应的就是性质2。现在主要是需要对第一项进行估计,显然它至少需要用到性质3与函数$\psi_x$。

但在第一项中,假设$A_n$可以准确地逼近$f_0$,那么第一项就相当于
$$|A_nf(x)-f(x)f_0(x)|=|A_nf(x)-f(x)|$$
就是需要证明的目标,似乎问题并没有变得更简单。这意味着上界的第二项并不是实质性的。

对于上界的第一项,假如我们可以对$f-f(x)f_0$进行估计(强调它的参数不必是$x$,于是函数为$f(y)-f(x)f_0(y)$),则利用保非负性,就可以对$A_nf(y)-f(x)A_nf_0(y)$进行估计,再取$y=x$,即得到目标。注意保非负性的实质是,$f_1\leq f_2\implies f_2-f_1\geq 0\implies A_n(f_2-f_1)\geq 0\implies A_nf_1\leq A_nf_2$(这里其实类似于单调性,因此也叫单调算子)。现在的问题就是对$f-f(x)f_0$进行估计,或者是对$f(y)-f(x)$进行估计,必然用到$\psi_x$,因为它就相当于一个二元函数。证明时先把它与伪距离函数联系起来,再取$y=x$。

下面的证明就是按照这个思路来的。
\begin{proof}

\end{proof}