\chapter{统计机器学习}
\section{降维}
\subsection{PCA}

\subsection{Kernel PCA}

\section{SVM}
\subsection{Optimal Margin分类}
考虑损失函数
\begin{empheq}{align}\label{optimal-margin-classification}
\min_{\btheta,\theta_0}\quad J&=\inv{2}\|\btheta\|^2+C\sum_{i=1}^{N}\mathcal{L}_{\rho}(y_i,\bx_i^T\btheta+\theta_0)\\
\mathcal{L}_{\rho}(y,\hat{y})&=\max\{0,\rho-y\hat{y}\}\\
y_i&\in\{-1,1\}\text{(二分类)}
\end{empheq}
损失函数中的第一项相当于Tikhonov正则化\ref{Tikhonov-regularization}。$\mathcal{L}_{\rho}$为Hinge损失函数。图像如下:
\pgfplotsset{
	myplot/.style={
		width=10cm, height=7.5cm,
		xlabel=$\hat{y}$, ylabel=$\mathcal{L}_{\rho}$,
		samples=50,
		xlabel style={at={(0.5,-0.08)}, anchor=south},
		ylabel style={rotate=-90, at={(0,0.5)}, anchor=east},
		legend style={draw=none, fill=none},
		xmin=-3, xmax=3,
        cycle list name=exotic
	}
}
\begin{center}
	\begin{tikzpicture}[>=stealth,
		every node/.style={rounded corners},
		declare function={
			hingelossp(\x)=(((1-\x)>0)*(1-\x));
            hingelossn(\x)=(((1+\x)>0)*(1+\x));
		}]
		
		\begin{axis}[myplot]
			\addplot+[mark=none, smooth, thick, domain=-3:3] {hingelossp(x)};
            \addplot+[mark=none, smooth, thick, domain=-3:3] {hingelossn(x)};
            \legend{$y=1$,$y=-1$};
		\end{axis}
	\end{tikzpicture}
\end{center}
从图中可以看出,如果$y_i=1$为正样本,则$\hat{y}$越大越好;反之,越小越好。不过这样可能会趋于无穷大或者无穷小,因此需要对$\btheta$进行约束,所以正则化项是必须的。

这个损失函数是不光滑的,利用松弛变量可以改变为一个约束优化问题:
\begin{empheq}{align}\label{optimal-margin-classification-slack}
\min_{\btheta,\theta_0,\bm{\xi}}\quad & J=\inv{2}\|\btheta\|^2+C\sum_{i=1}^{N}\xi_i\\
\text{s.t.}\quad & y_i(\bx_i^T\btheta+\theta_0)\geq \rho-\xi_i\\
&\xi_i\geq 0
\end{empheq}

注意看这个改写是怎么来的,取松弛变量$$\xi_i=\mathcal{L}_{\rho}(y_i,\bx_i^T\btheta+\theta_0)=\max\{0,\rho-y_i\hat{y}_i\}$$
显然$\xi_i\geq 0$,且$\xi_i\geq \rho-y_i\hat{y}_i$,分别对应第二个、第一个约束条件。

现在来说说这个Margin是什么意思。熟知给定一个超平面$\bx^T\btheta+\theta_0=0$,一个点$\bx_i$到平面的有符号距离是$\frac{\bx_i^T\btheta+\theta_0\bx_i}{\|\btheta\|}$,这是因为一个平面分空间为2部分,所以距离可以有符号,对应通常的上面、下面。如果给距离为正的点标识为正样本,取$y_i=1$,而负距离取$y_i=-1$,则$y_i(\bx_i^T\btheta+\theta_0\bx_i)$可以确保为非负值。

如果是线性可分的情形,则可以保证$y_i\hat{y}_i\geq 0$,此时对应$\xi_i=0$。如果是线性不可分,则有些点正样本的点可能距离为负,此时$y_i\hat{y}_i<0$,对应$\rho-\xi_i<0$,即$\xi_i\geq \rho>0$。因此优化任务就是希望$\xi_i$尽可能为负值。

所以这里的Margin就刻画了(但不是)点到超平面的距离。
\subsection{Maximum Margin分类——SVM}
\subsubsection{线性可分}
\paragraph*{问题表述}
对于上一节的问题\ref{optimal-margin-classification-slack},如果问题线性可分,则$\xi_i=0$,如果取$\rho=1$,则现在的问题是
\begin{empheq}{align}\label{svm-linear-seq-1}
\min_{\btheta,\theta_0}\quad & \inv{2}\|\btheta\|^2\\
\text{s.t.}\quad &y_i(\bx_i^T\btheta+\theta_0)\geq 1\\
\implies&\diag(\by)X\btheta+\theta_0\by\geq \bm{1}\\
\implies &\begin{bmatrix}
\diag(\by)X &\bm{0}\\\bm{0}&\by
\end{bmatrix}\begin{bmatrix}
\btheta\\
\theta_0\bm{1}_{N}
\end{bmatrix}\geq \bm{1}_{N}
\end{empheq}
这是一个线性线束二次规划问题。
\paragraph*{拉格朗日法}对于问题\ref{svm-linear-seq-1},其拉格朗日函数及对应的KKT条件为
\begin{empheq}{align}
L(\btheta,\theta_0,\bm{\lambda})&=\inv{2}\|\btheta\|^2-\sum_{i=1}^{N} \lambda_i(y_i(\bx_i^T\btheta+\theta_0)-1)\\
\pdv{L}{\btheta}&=\btheta-\sum_{i=1}^{N}\lambda_iy_i\bx_i=0\\
\pdv{L}{\theta_0}&=\bm{\lambda}^T\by=0\\
\lambda_i&(y_i(\bx_i^T\btheta+\theta_0)-1)=0\\
\lambda_i&\geq 0
\end{empheq}


\subsubsection{线性不可分}
\paragraph*{核函数的引入}

\section{树方法}

\subsection{核方法}
