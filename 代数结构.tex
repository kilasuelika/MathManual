\chapter{代数结构}

\section{代数结构一览}

\newpage
\thispagestyle{empty}
\newgeometry{left=1.5cm,bottom=0.4cm,top=0.5cm,right=1.5cm}
\begin{landscape}
\begin{tikzpicture}[grow  = right,
	sibling distance        = 6em,
	level distance          = 10em,
	edge from parent/.style = {draw, -latex},
	every node/.style       = {font=\footnotesize},
	sloped]
 \node [root] {集合}
child { node [env] {偏序集} 
		child { node [env] {格}
				edge from parent node [below] {交、并}
		}
		edge from parent node [above] {自反、传递} node[below]{反对称关系}
}
child { node [env] {半群}
		child { node [env] {幺半群}
				child { node [env] {群}
						child { node [env] {李群}
							edge from parent node [above] {解析流形} node [below] {$xy^{-1}$解析映射}
						}
						child { node [env] {交换群}
								child { node [env] {环}
										child { node (CR) [env] {除环}
												edge from parent node [above] {乘法逆元}
										}
										child { node [env] {交换环}
												child { node (F) [env] {域}
														edge from parent node  [above] {乘法逆元}
												}
												child { node [env] {$R$模}
														edge from parent node [above] {加法交换群$R$} node [below] {标量运算}	
												}
											edge from parent node [above] {乘法交换律}
										}
										edge from parent node [above] {乘法半群} node [below] {分配律}
								}
								edge from parent node [above] {(加法)交换律}
						}
						edge from parent node [above] {逆元}
				}
				edge from parent node [above] {单位元}
		}
		edge from parent node [above] {乘法封闭} node[below]{结合律} 
};
\draw[-latex] (CR) -- (F) node [pos=0.5,sloped,above] {乘法交换律};
\end{tikzpicture}

\end{landscape}
\newpage
\restoregeometry

\section{群}
\subsection{群的一般概念}
\subsubsection{群作用}
群作用是指群$G$作用于集合$\mathcal{S}$。假如对于$G$中任意元素$g$,存在$\alpha\in\mathcal{S}$,有$g\cdot \alpha \in \mathcal{S}$。则称群$G$作用于$S$。可以看出群作用其实相当于将集合$\mathcal{S}$映射为自身(或子集),即群$G$到$\mathcal{S}$的变换群内有一个同态映射$\eta$:
$$g^{\eta}\colon \alpha \rightarrow g\cdot \alpha$$
$\eta$称为群$G$在$\mathcal{S}$中的作用。

由于$\eta$是一个映射,自然地有不动点,在群论中称为核。对于$G$中某些元素,它们将$\mathcal{S}$中的元素$\alpha$映射成自身,称$G$中的这些元素为$\eta$的核。

\subsubsection{子群}
子群就是满足群性质的子集。

\paragraph{正规子群}正规子群是在共轭变换下不变的子群,即
\begin{definition}[正规子群]\label{normal-subgroup}
取$N$为一子群,若满足
$$\forall n\in N,g\in G,gng^{-1}\in N$$
则$N$是正规子群。
\end{definition}

\paragraph*{陪集}子群的完备扩张。
\begin{definition}[陪集]
	子群$H$关于$a$的左陪集为
	$$aH\coloneqq\{ax\mid x\in H\}$$
	类似地有右陪集。
\end{definition}
对于任意两个陪集$aH,bH$,它们要么相交,要么相等,因此一个群可分解为左陪集的交,或者右陪集的交,称为陪集分解。

\subsection{置换群与循环群}
\subsubsection{置换群的表示}通常用$\sigma,\pi$表示置换群(permutation),循环群是置换群表示的简化,本质上是一回事。

置换群用矩阵表示,比如
$$P=\begin{pmatrix}
1 & 2 & 3 & 4\\
1 & 3 & 2& 4
\end{pmatrix}=(1)(23)(4)$$

置换群矩阵的列表示,将第一行中的元素换成第二行中的元素,注意第一行的元素不是单纯地表示位置,而是表示特定的元素,所以交换列的顺序时,置换群是一样的。而循环群中的每个节是将前面的元素换成后面的,最后一个元素换成第一个。

\subsubsection{置换群的乘法}
注意置换群相乘时,是从右到左看的,按函数表示就是$\sigma(\pi(x))$,可以看出,内部的先算。举一个例子:
$$\begin{pmatrix}
1 & 2 & 3 &4\\
1 & 3 & 2 & 4
\end{pmatrix}\begin{pmatrix}
1 & 2 & 3 &4\\
2 & 1 & 3 &4
\end{pmatrix}=\begin{pmatrix}
1 & 2 & 3 &4\\
3 & 1 & 2 & 4
\end{pmatrix}$$

首先看第二个矩阵的第1列,表示将元素1换成2,而在第一个矩阵中,第二列将2换成3,所以1最终换成3。

\subsubsection{置换群的性质}
\paragraph*{阶数}首先转换成循环表示法, 会有若干个节,每节元素个数的最小公倍数,就是阶数。对于每个节来说,重复作用$n_i$次,会回到原来的状态,因此取最小公倍数,就表示所有节回到原来的状态。
\paragraph*{Cayley定理}每个群都与某个$\Sym(G)$的子群同构。$\Sym$是对称群,它由群$G$的所有permutation组成。也相当于说,每个群与某个置换群同构。

\subsection{商群}
商群这个概念很重要、基本,但也有点难以理解。 它相当于一个群配上等价关系:
\begin{definition}[商群]
给定群$G$和它的正规子群$N$,定义商群$G/N$为$N$在$G$中的所有左陪集,或者:
$$G/N\coloneqq \{aN\in G  \}$$
商群中的乘法定义为
$$(aN)(bN)\coloneqq (ab)N$$
由于$N$是正规子群,所以取$aN$或者$Na$是一回事。
\end{definition}

\subsection{群的例子}
\subsubsection{线性群}
\begin{longtable}{ccc}
	\toprule
	表示& 含义 &维数\\
	\midrule
	$\GL(n,F)$ &$F$上全体可逆矩阵& $n^2$\\
	$\SL(n,F)$ &$F$上全体行列式为1的可逆矩阵&$n^2-1$\\
	$\text{O}(n,F)$ & $\GL(n,F)$中的正交矩阵&\\
	$\text{SO}(n,F)$ & $\SL(n,F)$中的正交矩阵&\\
	\bottomrule
\end{longtable}

\section{环}

\section{域}

\section{特殊代数结构的例子}

\subsection{Tropical Geometry}
属于代数几何的一种.
\begin{definition}{Tropical Geometry}{}
\begin{empheq}{align*}
x\oplus y&=\max (x,y)\\
x\odot y&=x+y\\
x^a&=ax\\
\bx^{\bm{a}}&=c\odot x_1^{a_1}\odot\cdots\odot x_d^{a_d}\mtag{Tropical monomial}\\
f(\bx)&=c_1\bx^{\bm{\alpha}_1}\oplus\cdots\oplus\bx^{\bm{\alpha_r}}\mtag{Tropical Polynomial}\\
f(x)\oslash g(x)&=f(x)-g(x)\mtag{Rational Function}
\end{empheq}

\end{definition}


前两个算子可以构成半环$\mathbb{T}\coloneqq (\mathbb{R}\cup\{-\infty\},\oplus,\odot)$.这里的$\max$也可以换成$\min$,两者等价.
