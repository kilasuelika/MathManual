\subsection{Banach不动点定理}
核心是压缩映射:压缩映射有且仅有一个不动点.

\begin{theorem}{Banach不动点定理}{}
设$(X,d)$为完备的距离空间.则任何压缩映射$f:X\rightarrow X$有且仅有一个不动点$x\in X$.

此外,任意给定点$x_0\in X$,由
$$x_{n+1}=f(x_n),\ n\geq 0$$

定义的序列$\{x_n\}_{n=0}^\infty $当$n\rightarrow 0$时,收敛于$x$,且成立以下估计:
$$\parallel x_n-x\parallel \leq Ck^n,\ n\geq 0,\ C\coloneqq \frac{d(f(x_0),x_0)}{1-k}$$

\end{theorem}

证明过程依赖于Cauchy列.

\begin{proof}
(柯西列)对任何$p\geq 1$,
$$d(x_{p+1},x_p)\leq kd(x_p,x_{p+1})\leq\cdots\leq k^pd(x_1,x_0)$$
因此对任意$m>n\geq 0$,
\begin{empheq}{align*}
d(x_m,x_n)&leq\sum_{p=n}^{m-1}d(x_{p+1},x_p)\leq\left(\sum_{p=n}^{m-1}k^p\right)d(x_1,x_0)\\
&\leq k^n\left(\sum_{p=0}^{m-n-1}k^p\right)d(x_1,x_0)\leq \frac{k^n}{1-k}d(x_1,x_0)
\end{empheq}

(存在不动点)空间$(X,d)$是完备的,所以存在$x\in X$,使得$\lim_{n\rightarrow \infty}x_{n+1}=x$.因为压缩映射显然是连续的,所以
$$f(x)=\lim_{n\rightarrow \infty}f(x_{n})=\lim_{n\rightarrow \infty}x_{n+1}=x$$
因此$x$是不动点.

(不动点惟一)假设$y\in X$也是不动点,则
$$d(x,y)=d(f(x),f(y))\leq kd(x,y)$$

所以$y=x$,因此不动点惟一.
	
\end{proof}

利用序列$\{x_n\}_{n=0}^\infty $逼近不动点的办法也叫逐次逼近或者Picard方法.